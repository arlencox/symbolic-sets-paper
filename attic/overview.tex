\section{Overview}
\label{s:2:over}
In this section, we present two static analyses that make use of set
reasoning, in order to compute high level semantic properties of
programs.
These analyses rely on abstract interpretation~\cite{cc:popl:77} and
on an abstraction of program states, that describes data structures
and their contents.
An abstract domain defines a set of predicates that an analysis may
use, as well as operators to over-approximate the effect of program
behaviors on these predicates, and their implementation.

\paragraph{Inferrence of properties of open objects.}
Dynamic programming languages such as JavaScript feature {\em Open objects}
that support dynamic addition and deletion of attributes and iteration over
them.
The analysis presented in~\cite{hoo:14:sas} computes relations among
objects, so as to verify programs such as a copy of the attributes of an
object into another object.
To achieve this, it needs to infer relations between the sets of attributes
of distinct objects.
Objects have an unbounded number of attributes, thus the analysis requires
some abstraction over the attributes and their contents.
\begin{figure}[t]
  \newcommand{\picscale}{1}
  \begin{center}
    \subfigure[A concrete state]{ \label{f:1a:hoo:conc}
      \tikzpics{\picscale}{hoo-conc}
    }
    \quad
    \subfigure[Corresponding abstract state]{ \label{f:1b:hoo:abs}
      \tikzpics{\picscale}{hoo-abs}
    }
  \end{center}
  \caption{Open objects and their abstraction}
  \label{f:1:hoo}
\end{figure}
Figure~\ref{f:1a:hoo:conc} represents a very simplified state, after
the copy of the contents of objects \( \vary \) into object \( \varx \).
This picture does not show field contents which are ignored here. % end !!!
Instead, we focus on the set of attributes each object comprises since
taking contents into account would not change much the demonstration.
To abstract precisely the relations between the attributes of both objects
(\ie, in this case, all attributes of \( \vary \) are also attributes of
\( \varx \)), we need to describe the fields of each object as the union
of a series of sets of attributes, and to carry out relations over these
sets.
Figure~\ref{f:1b:hoo:abs} depicts such an abstract state, where \( X, Y \)
stand for the set of all attributes of objects \( \varx, \vary \).
The set of fields of \( \varx \) is partitioned into two subsets \( X_0,
X_1 \), and \( X_0 \) is exactly equal to the set of all the attributes
of \( \vary \).
Moreover, to infer invariants such as the abstract state shown in
Figure~\ref{f:1b:hoo:abs}, the analysis needs to compute both the
object partitions and the associated set relations, in the same time
as those depend on each other.
When dealing with updates, the analysis {\em expands} the partitions and the
corresponding set relations.
Conversely, the fixpoint computation causes the folding of partitions,
which requires the computation of {\em compact disjunctions} of set
properties.
Last, the verification of a post-condition requires the {\em prooving}
of set properties under a set of assumptions (for instance, that \( Y
\subseteq X \), under the abstract state of Figure~\ref{f:1b:hoo:abs}).

\paragraph{Shape analysis in presence of unstructured sharing.}
We now discuss a second example of static analysis that needs to perform
set reasoning.
The shape analysis for data-structures with unbounded sharing presented
in~\cite{memcad:15:sas} relies on separation logic~\cite{r:lics:02} to
describe memory states, and on inductive definitions to summarize
unbounded structures such as lists.
Unstructured sharing is very challenging as it cannot be described using
conventional inductive definitions.
Figure~\ref{f:2a:memcad:conc} displays the representation of a three nodes
graph using an adjacency list data-structure.
To summarize such a structure using inductive predicates in separation
logic, \cite{memcad:15:sas} proposes to augment the list inductive
predicates with set information, which express where edges may point to.
This representation is shown in Figure~\ref{f:2b:memcad:sum} in a form
where the first node is kept materialized.
It both asserts that edges of that node as well as edges from other nodes
point to the address of a valid node, namely an element of \( \{ \naddr{0}
\} \suplus \mcE \).
\begin{figure}[t]
  \newcommand{\picscale}{1}
  \begin{center}
    \subfigure[A concrete state]{ \label{f:2a:memcad:conc}
      \tikzpics{\picscale}{memcad-conc}
    }
    \quad
    \subfigure[Summarization principle]{ \label{f:2b:memcad:sum}
      \tikzpics{\picscale}{memcad-abs}
    }
  \end{center}
  \caption{Summarization of an adjacency list-based graph representation}
  \label{f:2:memcad}
\end{figure}
Therefore, the analysis needs to track the set information adjoined to
the structural inductive definitions.
When a node or edge is materialized, the set of graph relations should
be extended so as to reflect the knowledge about its successor(s).
Conversely, the folding of inductive predicates necessitates the
verification that some set relations hold in the abstract level.

\paragraph{Need for set reasoning.}
Both analyses consist of a composite abstract domain, that comprises a
{\em structural abstract domain} describing the shape of structures, and
of a {\em set abstract domain} expressing relations among set and value
(pointers, field names, contents...) entities.
The operations that need to be performed in the set abstract domain are
also similar, and mainly include the meet with additional constraints,
the generalization of constraints, and the entailment checking.

