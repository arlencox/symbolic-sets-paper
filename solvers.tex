\section{Solver-based Abstractions}
\label{sec:solvers}

Because one of the core components of set abstraction is the Boolean algebra, it is possible to construct abstract domains from off-the-shelf satisfiability solvers.  The construction relies upon the fact that the standard Boolean algebra is a finite height lattice ordered by implication.  This means that no specific invariant generation procedure is required.

The syntax of the abstraction is the standard Boolean algebra with existential quantification.  There are two reasons this is a good logic to use.  First, it is a fairly well supported logic for which there are efficient solvers.  Second, it remains finite height and thus needs no specialized invariant generation procedure as would be required with a set logic such as BAPA~\cite{knr:jar:06}.  Since cardinality is not a key requirement for symbolic sets, the analysis can often be sufficient without it.

Domain operations are translated into Boolean algebra formulas.  $\adomjoin(\astate_1,\astate_2)$ translates into $\astate_1 \vee \astate_2$.  $\adomassume(\astate, L)$ translates into $\astate \wedge \textrm{conv}(L)$, assuming that $\textrm{conv}(L)$ converts the constraint $L$ into its Boolean algebra equivalent as in Section~\ref{s:4:3:bdd}.  $\adomforget(\astate, X)$ translates into $\exists X. \; \astate$.  These are accumulated across the whole analysis and thus may grow arbitrarily deep.  It is possible that on-the-fly simplification could be used, but we elect to use whatever internal functionality is provided by the solver (in this case Z3~\cite{mb:tacas:08}).

Query operations are translated into solver queries.  $\adomisle(\astate_1, \astate_2)$ translates to $\textsc{Valid}(\astate_1 \rightarrow \astate_2)$.  This is implemented incrementally by conditionally adding constraints for each query and checking satisfiability under assumptions.  The $\adomisbot(\astate)$ query translates into $\textsc{Valid}(\neg \astate)$.

\begin{example}[Solver-based abstraction operations]
Domain operations accumulate constraints, so simplification is performed by the solver when a query happens.  In the following sequence, there are no queries, so constraints only accumulate.
\begin{align*}
  \astate_0 &= \top & &= \textsc{True} \\
  \astate_1 &= \adomassume(\astate_0, X \subseteq Y \wedge Y \subseteq Z) & &= \textsc{True} \wedge X \rightarrow Y \wedge Y \rightarrow Z \\
  \astate_2 &= \adomforget(\astate_1, Y) & &= \exists Y. \; \textsc{True} \wedge X \rightarrow Y \wedge Y \rightarrow Z
\end{align*}
If the query $\adomprove(\astate_2, X \subseteq Z)$ is performed, the following check is made: $\textsc{Valid}((\exists Y. \; \textsc{True} \wedge X \rightarrow Y \wedge Y \rightarrow Z) \rightarrow (X \rightarrow Z))$.  This holds trivially.
\end{example}
