\section{Constructed Set Abstractions}
\label{sec:constructed}
An abstract domain is defined by a class of set constraints and their machine
representation, and abstract operations following the signature given in
Section~\ref{s:3:2:sign}.
In this section, we introduce three basic set abstract domains (respectively
based on linear constraints, QUIC graphs, and BDDs) and two set abstract
domain functors, that map a set domain into another, more expressive or
efficient one.

\subsection{Linear Set Constraints}
\label{s:4:1:lin}
% - set of predicates
% - concretization function
% - main algorithm principle: lazy, quasi normalization
% - example (from the MemCAD test cases

\subsection{QUIC graphs}
\label{s:4:2:quic}
% - set of predicates
% - concretization function
% - algorithms main ideas
% - example

\subsection{BDD-based Set Constraints}
\label{s:4:3:bdd}
% - BDD structure quickly explained
% - encoding of a class of set constraints (it gives the gamma)
% - algorithms main ideas
% - example

\subsection{The Equalities Domain Functor: Compact Equality Constraints}
\label{s:4:4:eqs}
% - origin of the equalities problem (cases where there are too many variables
%   and most of them are equal, consequences in terms of complexity)
% - principle of the functor
% - abstract states and concretization
% - example

\subsection{The Packing Domain Functor: Sparse Constraints}
\label{s:4:5:packs}
% - issue with relational abstract domains
% - packing principle (citations to Astree)
% - abstract states and concretization
% - example