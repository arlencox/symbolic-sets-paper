\section{Constructed Set Abstractions}
\label{sec:constructed} \label{s:4:domains}
An abstract domain is defined by a class of set constraints, their machine
representation, and the abstract operations following the signatures given in
Section~\ref{s:3:abs}.
In this section, we introduce three basic set abstract domains (respectively
based on linear constraints, QUIC graphs, and BDDs) and two set abstract
domain functors, that lift a set domain into another, more expressive or
efficient one.

\subsection{Linear Set Constraints}
\label{s:4:1:lin}
\newcommand{\adomlin}{\adom_{\zcl}}
\newcommand{\gammalin}{\gamma_{\zcl}}
\paragraph{Abstract elements and their concretization.}
Our first set abstract domain relies on {\em linear} set equality
constraints, of the form \( \aform{\setvx = \{ y_0, \ldots, y_k \} \suplus
  \setvz_0 \suplus \ldots \suplus \setvz_l} \).
The advantage of such constraints is to provide a rather straightforward
normalization of the representation of constraints.
Note they also include emptiness constraints.
Our implementation of abstract domain \( \adomlin \) describes three kinds
of constraints:
\begin{compactitem}
\item acyclic {\em linear} constraints of the form \( \aform{\setvx =
    \setvy_0 \suplus \ldots \suplus \setvy_k \suplus \setvz_0 \suplus \ldots
    \suplus \setvz_l} \), where \( \setvy_0, \ldots, \setvy_k \) are
  singletons (in our implementation, each variable may appear at most
  {\em once} as the left-hand side of such a constraint, to enable
  normalization);
\item inclusion constraints of the form \( \aform{\setvy \subseteq \setvx} \);
\item equality constraints of the form \( \aform{\setvy = \setvx} \).
\end{compactitem}
Thus, an element of \( \adomlin \) is either \( \bot \) or a conjunction of
such constraints.
The associated concretization \( \gammalin: \adomlin \rightarrow
\partsof{\states} \) is of the same form as that of the symbolic sets
language of Definition~\ref{d:1:symsets} (thus, we do not formalize it
in full details).
The machine representation utilizes persistent dictionaries, that stand
for functions over a finite domain.
This reduces basic queries for facts (such as, ``does abstract state
\( \astate \) entail that \( \setvx \subseteq \setvy \suplus \setvz
\)~?'') to dictionary searches.

\paragraph{Abstract operators.}
The core algorithm of \( \adomlin \) normalizes abstract values by
expanding nested linear constraints.
For instance, \( \aform{\setvx_0 = \setvx_1 \suplus \setvx_2 \wedge
  \setvx_1 = \setvx_3 \suplus \setvx_4} \) is rewritten into
\( \aform{\setvx_0 = \setvx_2 \suplus \setvx_3 \suplus \setvx_4 \wedge
  \setvx_1 = \setvx_3 \suplus \setvx_4} \) at the machine representation
level.
This process terminates as constraints represented in \( \adomlin \)
do not contain cycles.
It is performed incrementally by all abstract operations.

Abstract operations \( \adomisbot, \adomassume, \adomprove \) are all
made very fast by this normalization.
Operation \( \adomforget \) simply drops all constraints that involve
a given set variable.
Finally, \( \adomjoin \) and \( \adomwiden \) need to {\em generalize}
constraints.
\begin{example}
  Let us assume that \( \astate_0 \) (\resp, \( \astate_1 \)) stands for
  the set of constraints \( \aform{\setvx_0 = \setvx_1 \suplus \setvx_2
    \wedge \setvx_3 = \emptyset} \) (\resp, \( \aform{\setvx_0 = \setvx_1
    \suplus \setvx_2 \suplus \setvx_3} \)).
  Then \( \adomjoin( \astate_0, \astate_1 ) \) returns an element that
  represents the constraint \( \aform{\setvx_0 = \setvx_1 \suplus \setvx_2
    \suplus \setvx_3} \).
\end{example}
\memcad~\cite{memcad:15:sas} relies on \( \adomlin \) to represent set
constraints since it mainly needs to express constraints over set
partitions.
On the other hand, \( \adomlin \) is not adapted to the precise
description of non disjoint unions.

\subsection{QUIC graphs}
\label{s:4:2:quic}

A QUIC graph~\cite{ab:ecoop:13} is a directed hypergraph data structure used to represent relational set constraints.  Each edge in the hypergraph corresponds to a subset constrtaint and each hypergraph is a conjunction of subset constraints where each constraint is of the form $\aform{X_1 \cap \ldots \cap X_n \subseteq Y_1 \cup \ldots \cup Y_m}$.  Each variable can also be constrained to be a singleton, with constraints such as $\aform{|X| = 1}$.  The concretization $\gamma_q : \adom_q \rightarrow \partsof{\states}$ is of the same form as that of the symbolic sets language of Definition~\ref{d:1:symsets}.

% - algorithms main ideas
QUIC graphs are designed for efficiently performing two operations:
(1) $\adomforget$, which matches edges containing the symbol to be
forgotten with each other to produce new edges without that symbol; and
(2) content reasoning, which is not a design goal for symbolic sets.
The $\adomjoin$ and $\adomwiden$ operations are primarily based on
saturation heuristics.
They keep common conjunctions from both arguments.
To aid this process, they use a form of saturation that produces new
conjuncts based on pattern matches.
A sufficiently large set of patterns must be provided to attain precision,
but additional patterns increase the cost of joins.
% - example
\begin{example}[QUIC graph join] \label{ex:Qjoin}  Consider the following join operation:
    \begin{align*}
      \astate_0 &= \aform{X_1 \subseteq X_3 \logand X_3 \subseteq X_2} &
      \astate_1 &= \aform{X_1 \subseteq X_4 \logand X_4 \subseteq X_2} &
      \adomjoin(\astate_0, \astate_1 )
    \end{align*}
    There is an obvious result: $\aform{X_1 \subseteq X_2}$.  Whether or not QUIC graphs derive this result or $\aform{\top}$ is determined by the pattern matches that are installed.  If the pattern that takes $\aform{X_a \subseteq X_b \logand X_b \subseteq X_c}$ and generates $\aform{X_a \subseteq X_c}$ is used, the pattern will be applied to both sides and then common conjuncts kept, getting the desired result.  Without that pattern or a similar substitute, QUIC graphs derive $\aform{\top}$.
\end{example}

\subsection{BDD-based Set Constraints}
\label{s:4:3:bdd}
% - BDD structure quickly explained
Binary decision diagrams (BDDs)~\cite{bdd:trans:86} are a canonical representation of Boolean algebraic functions.  There are three basic syntactic elements of a BDD.  The $\textsc{True}$ and $\textsc{False}$ elements represent the obvious constants, but $\textrm{ITE}(X,B_t,B_e)$ is an if-then-else structure.  If the variable $X$ is $\true$, the result of evaluating $B_t$ is returned, otherwise the result of evaluating $B_e$ is returned.
\begin{align*}
  B ::= \textsc{True} \ | \ \textsc{False} \ | \ \textrm{ITE}(X,B_t,B_e)
\end{align*}
What makes BDDs canonical is that we only consider reduced, ordered BDDs, where it is assumed that there is a total order $\prec$ on the variables.  An $\textrm{ITE}(X,B_t,B_e)$ can only be constructed if $X \prec X'$ for all variables $X'$ in $B_t$ or $B_e$.  Additionally, structural sharing is mandated, so the reuse of the same syntax is referentially identical to any other use of that syntax.

% - encoding of a class of set constraints (it gives the gamma)
The encoding of constraints maps operators from their constraint form (as in Definition~\ref{d:1:symsets}) to their Boolean algebraic form: $\cup \mapsto \vee, \ \cap \mapsto \wedge, \ \comp{} \mapsto \neg, \ \subseteq \mapsto \rightarrow, \ = \mapsto \leftrightarrow$. After singleton set constraints are dropped, the remaining constraints are directly and exactly represented by the BDD.  

% - algorithms main ideas
Domain operations are straightforward: $\adomjoin$ and $\adomwiden$ are implemented with the $\vee$ operation, which is precise and does not need any rules or heuristics; $\adomforget$ takes advantage of reasonably efficient quantifier elimination provided by BDDs and uses existential quantifier elimination to drop variables.  Queries such as $\adomisle$ are easily implemented using validity checking functionality provided by BDDs.  Critically, because BDDs are a canonical form, many operations such as $\adomforget$ and $\adomassume$ become much more efficient, whereas the operation $\adomisbot$ becomes an $O(1)$ check.

% - example
\begin{example}[BDD-based join]  Consider the same inputs as Example~\ref{ex:Qjoin}.  Encoding them to BDDs (and using some Boolean-algebraic notation as shorthand) yields the following results:
\begin{align*}
  & \astate_0 = \aform{X_1 \subseteq X_3 \logand X_3 \subseteq X_2} =
  \textrm{ITE}(X_1, X_2 \wedge X_3, \textrm{ITE}(X_2, \textsc{True},
  \neg X_3))
  \\
  & \astate_1 = \aform{X_1 \subseteq X_4 \logand X_4 \subseteq X_2} =
  \textrm{ITE}(X_1, X_2 \wedge X_4, \textrm{ITE}(X_2, \textsc{True},
  \neg X_4))
  \\
  & \adomjoin(\astate_0, \astate_1 ) =
  \textrm{ITE}(X_1, X_2 \wedge \textrm{ITE}(X_3, \textsc{True}, X_4),
  \\
  & \qquad \qquad \qquad \qquad \qquad \quad \;\;
  \textrm{ITE}(X_2, \textsc{True}, \textrm{ITE}(X_3, \neg X_4, \textsc{True})))
\end{align*}
  The result of this join is equivalent to the set constraints $\aform{X_1 \subseteq X_2}$, $\aform{X_1 \subseteq X_3 \cup X_4}$, and $\aform{X_3 \cap X_4 \subseteq X_2}$, which includes not only the obvious result of $\aform{X_1 \subseteq X_2}$, but also other, possibly useful results.  It is a precise join.
\end{example}

We implement the BDD abstraction on top of the CU decision diagrams package~\cite{cudd}, which is high performance and offers the ability to extract prime implicants (as in~\cite{prime:mit:92}).  The prime implicants of the negation of the Boolean function are easily converted to conjuncts of the form used by QUIC graphs.

\subsection{The Equalities Domain Functor: Compact Equality Constraints}
\label{s:4:4:eqs}
\newcommand{\eqrep}{\ensuremath{Q}}
% - origin of the equalities problem (cases where there are too many variables
%   and most of them are equal, consequences in terms of complexity)

When analyzing real programs, in addition to complex set constraints, there are often many very simple equality constraints of the form $\aform{X = Y}$.  These can be a problem in several ways.  For example, equalities are normalized and handled precisely in BDDs, but they can grow the size of the representation significantly.  This results in significantly increased memory usage and decreased efficiency since many BDD operations rebuild the BDD.  In QUIC graphs, equalities grow the size of the graph, and place significantly more load on the pattern matching system, potentially causing an explosion in the number of constraints.  This is because QUIC graphs can represent each variant of an expression rewritten using all available equalities.  In linear set abstractions, there are similar potential problems.

As a result, abstractions like QUIC graphs and the linear set abstraction have special handling for equality.  This improves performance and precision at the cost of complexity.  Instead, much of this complexity can be moved outside the abstraction and handled by lifting the abstraction to one that keeps track of equalities separately from other kinds of constraints.

% - principle of the functor
The equality functor serves as an intermediary between the domain interface and the abstract domain that is being lifted.  It intercepts equality constraints and handles them externally, preventing them from being seen by the underlying abstract domain.  This saves the domain from the cost and complexity of handling the equalities.

% - abstract states and concretization
The equality functor defines a set of equivalence classes $\eqrep$.  The set of equivalence classes is a map $\setvars \rightarrow \setvars$ that maps each variable to the chosen representative for the equivalence class.  The functor then lifts an abstract state $\adom$ into a tuple $(\adom, \eqrep)$.  In the lifting, $\adom$ is restricted to only have symbols that are representatives for the equivalence class.  Therefore, when an equality is added that merges two equivalence classes, the resulting representative replaces the two previous representatives in $\adom$.

The concretization ensures that all symbols in the same equivalence class map to the same concrete set:
\begin{align*}
\gamma((\eqrep,\adom)) = \{ \ \state \ | \ \state \in \gamma(\adom) \wedge \forall X,Y \in \setvars^2. \; \eqrep(X) = \eqrep(Y) \rightarrow \state(X) = \state(Y) \ \}
\end{align*}

Domain operations $\adomjoin$, $\adomwiden$, and $\adomisle$ unify their corresponding $\eqrep$s, pushing any non-common equalities into the underlying domain.  This ensures that the underlying domain determines the precision, but it is not required to handle most of the load of the equalities.  The $\adomassume$ operation rewrites the constraint, extracting the equalities and rewriting remaining variables to their representatives before passing the constraint to the underlying domain.
 

% - example
\begin{example}[Equality functor join]
Consider the following two abstract states, where the underlying domain is just shown as symbolic set constraints:
\begin{align*}
  \astate_0 = ([W \mapsto W, X \mapsto W, Y \mapsto W], \aform{W \subseteq Z}) \ \ 
  \astate_1 = ([X \mapsto X, Y \mapsto X], \aform{W \subseteq X \wedge X \subseteq Z})
\end{align*}
In the join, the equivalence classes are unified, producing the resulting $\eqrep$: $[X \mapsto X, Y \mapsto X]$.  The equality $\aform{W = X}$ from $\astate_0$ is not represented in the unification, so it is added back to the underlying domain in $\astate_0$.  The result is therefore
\begin{align*}
  ([X \mapsto X, Y \mapsto X], \adomjoin(\aform{W = X \wedge W \subseteq Z}, \aform{W \subseteq X \wedge X \subseteq Z}))
\end{align*}
\end{example}

\subsection{The Packing Domain Functor: Sparse Constraints}
\label{s:4:5:packs}
% - issue with relational abstract domains
Most relational domains have a complexity that is related to the number of variables constrained by the abstract state.  For example, BDDs, in the worst case, are exponential in the number of variables.  However, in many programs, there are relatively small clusters of variables that are related.  Therefore it is possible to increase the efficiency of an analysis by representing each cluster of variables by a separate abstract state~\cite{ens:pldi:03}.

% - packing principle (citations to Astree)
If each of $m$ clusters of $n$ variables is represented by a separate abstract state, rather than operations having a complexity of, for example, $O(2^{m\cdot n})$, they can have complexity $O(m\cdot 2^n)$.  To do this, all variables are initially assumed to be in their own cluster.  Clusters are merged whenever variables from each cluster occur in the same constraint.  In this way the clusters are dynamically determined, which is required when an abstract domain is used as a library and thus a pre-analysis cannot be performed.

% - abstract states and concretization
An abstract state in the packing functor consists of one of three values: $\top$, $\bot$, or a map $M : \#_M \rightarrow \adom$ that maps cluster ids in $\#_M$ to abstract states from the domain being lifted.  The $\top$ and $\bot$ values concretize as they do in Figure~\ref{f:4:symsets}.  The map concretizes as follows:
\begin{align*}
  \gamma(M) = \{ \state \ | \ \forall \astate \in \textrm{Range}(M). \; \state \in \gamma(\astate) \}
\end{align*}

% - example
\begin{example}[Constraining a packed abstract state]
Consider the following abstract state represented by the logic from Definition~\ref{d:1:symsets}, lifted into two packs with ids $0$ and $1$:
$
  \astate_0 = [0 \mapsto \aform{X_0 \subseteq X_1}; \; 1 \mapsto \aform{Y_0 \subseteq Y_1} ]
$.
The operation $\adomassume(\astate_0, Y_1 \subseteq Y_2)$ operates only on pack id $1$.  It does not have to involve any computation on pack $0$.  The resulting pack $1$ is: $1 \mapsto \adomassume(\aform{Y_0 \subseteq Y_1}, Y_1 \subseteq Y_2)$.
\end{example}
