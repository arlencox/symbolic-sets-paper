\section{Logic and Set Abstraction}
\label{sec:logic-and-set-abstraction} \label{s:3:abs}
We now define the elements and operators of a set abstract domain that meets the needs of all the analyses shown in Section~\ref{s:2:over}.

\paragraph{Concrete states.}
In this paper, we use symbols $\setvw$, $\setvx$, $\setvy$, and $\setvz$ as set variables
and let $\setvars$ represent the set of all such variables.
We are interested in purely symbolic set relations, and do not make any
assumption on the type of the set elements (in practice these are pointers
or scalars).
We let \( \values \) denote the set of all these elements.
A concrete state is a function \( \state: \setvars \rightarrow
\partsof{\values} \).
We write \( \states \) for the set of such elements.

\paragraph{Symbolic sets.}
Before we set up the signature of abstract domains, we fix a language
of set predicates, that will be used as a basis for abstract elements,
and for the communication with the set abstract domain.
\begin{definition}[Symbolic Sets]
  \label{d:1:symsets}
  {\em Symbolic sets} are defined by the grammar:
  \begin{align*}
    L (\in \symsets) ::=
    & \ L \wedge L \
    | \ E \subseteq E \
    | \ \card{X} = 1 \
    | \ \top \
    | \ \bot
    & E ::=
    & \ \emptyset \ | \ X \ | \ \comp{E} \ | \ E \cup E \ | \ E \suplus E
  \end{align*}
\end{definition}
The meaning of these contraints is straightforward, but we give a formal
definition in Figure~\ref{f:4:symsets} for clarity.
A model of a set expression $E$ is a concrete state $\state$ and a set
of concrete values $c$.
A model of a logical expression $L$ is a concrete state $\state$.
The concretization is $\gamma(L) = \{\ \sigma \ |\  \sigma \models L\ \}$
and we use $\aform{L}$ for abstract states with the same concretization.
\begin{figure}[t]
  \begin{align*}
    & \state, c \models \emptyset \textrm{ iff } c = \emptyset
    \qquad
    \state, c \models X \textrm{ iff } c = \state(X)
    \qquad \state, c \models \comp{E}
    \textrm{ iff }
    \state, c' \models{E}
    \textrm{ and } \forall v \in \values.
    \; v \in c \Leftrightarrow v \not\in c'
    \\
    & \state, c \models E_1 \cup E_2
    \textrm{ iff }
    \state, c_1 \models E_1
    \textrm{ and } \state, c_2 \models E_2
    \textrm{ and }
    \forall v \in \values. \; v \in c \Leftrightarrow v \in c_1 \vee
    v \in c_2
    \\
    & \state, c \models E_1 \suplus E_2
    \textrm{ iff }
    \state, c_1 \models E_1
    \textrm{ and } \state, c_2 \models E_2
    \textrm{ and }
    \forall v \in \values. \; v \in c \Leftrightarrow v \in c_1 \vee
    v \in c_2
    \textrm{ and } c_1 \cap c_2 = \emptyset
    \\
    & \state \models L_1 \wedge L_2
    \textrm{ iff }
    \state \models L_1 \textrm{ and } \state \models L_2
    \qquad
    \state \models \card{E} = 1
    \textrm{ iff }
    \state, c \models E \textrm{ and } \exists v \in \values. \; c = \{ v \}
    \\
    & \state \models E_1 \subseteq E_2
    \textrm{ iff }
    \state, c_1 \models E_1 \textrm{ and } \state, c_2 \models E_2
    \textrm{ and } \forall v \in \values. \; v \in c_1 \rightarrow v \in c_2
    \qquad \state \models \top
    \qquad \state \not\models \bot
  \end{align*}
  \caption{Symbolic set constraint language}
  \label{f:4:symsets}
\end{figure}
We shall also use the following derived logical forms for simplicity:
\[
E_1 \cap E_2 \defeq \comp{(\comp{E_1} \cup \comp{E_2})}
\qquad
E_1 = E_2 \defeq E_1 \subseteq E_2 \wedge E_2 \subseteq E_1
\qquad
E_1 \setminus E_2 \defeq E_1 \cap \comp{E_2}
\]

\paragraph{Set abstraction.}
A {\em set abstract domain} is defined by a set of {\em abstract elements}
\( \adom \) which describe the family of logical properties it can
express and a concretization function \( \gammadom: \adom \rightarrow
\partsof{\states} \) that maps each element of \( \adom \) into the set
of concrete states that satisfy it.
Abstract elements are characterized by
(1) the symbolic sets they describe and
(2) their machine representation.
The latter is usually very different from the formulas, and will be
discussed in Section~\ref{s:4:domains}.
\begin{example}[(Non-)Emptiness set domain]
  \label{ex:1:mt}
  A very basic example of such a domain is the {\em (non-)emptiness} domain
  that comprises the following elements:
  \begin{compactitem}
  \item \( \bot \), which denotes the unsatisfiable abstract constraint
    (\ie, \( \gammadom( \bot ) = \emptyset ) \));
  \item the functions from \( \setvars \) into \( \{ [=\emptyset],
    [\not=\emptyset], \top \} \), which map each set variable into
    its emptiness value.
  \end{compactitem}
  For instance, \( \{ \setvx \mapsto \top; \setvy \mapsto [=\emptyset] \} \)
  stands for \( \aform{\setvy \subseteq \emptyset} \) and concretizes into
  \( \gamma( \setvy \subseteq \emptyset ) \).
\end{example}

\paragraph{Operations over Set Abstractions.}
We now formalize the main operations and logical elements needed so that
we can use a set abstract \( \adom \) domain for either of the static
analyses shown in Section~\ref{s:2:over}.
\begin{asparaitem}
\item \emph{Basic logical elements.}
  Static analyses typically start with an unconstrained state.
  This is indicated by a \( \adomtop \in \adom \) element with full
  concretization, \ie, \( \gammadom( \adomtop ) = \states \).
  Similarly, the abstract element \( \adombot \in \adom \) should describe
  the unsatisfiable abstract constraint (\ie, \( \gammadom( \adombot ) =
  \emptyset \)).
  In Example~\ref{ex:1:mt}, \( \adombot \) is \( \bot \) and \( \adomtop \)
  is \( \lambda (x \in \setvars) \cdot \top \).
  Moreover, a static analysis often has to determine if an abstract state
  describes unsatisfiable constraints.
  Thus, \( \adom \) should provide an operator \( \adomisbot: \adom
  \rightarrow \{ \true, \false \} \) such that \( \adomisbot( \astate )
  = \true \Longrightarrow \gammadom( \astate ) = \emptyset \).

\item \emph{Forgetting a set variable.}
  Static analysis tools drop set variables that become redundant.
  In the open object example of Section~\ref{s:2:over}, this occurs when
  the singleton symbol is eliminated at the end of the loop.
  To do this, we require the set abstract domain \( \adom \) to provide
  an operator \( \adomforget: \adom \times \setvars \rightarrow \adom \)
  that discards a symbol from the abstract state.
  % forget soundness property

\item \emph{Assuming set constraints.}
  As noted in Section~\ref{s:2:over}, an important set reasoning step
  {\em restricts an abstract state with set constraints}, thus set
  domain \( \adom \) should provide an operator \( \adomassume: \adom
  \times \symsets \rightarrow \adom \), which conservatively represents
  a constraint into an abstract state, \ie ensures that, for all
  \( \astate, L \), \( \gammadom( \astate ) \cap \gamma( L ) \subseteq
  \gammadom( \adomassume( \astate, L ) ) \).
  Note that this operator also makes use of the symbolic set language
  of Definition~\ref{d:1:symsets} in order to describe constraints
  communicated to the domain.
  
\item \emph{Verifying set constraints.}
  Similarly, set reasoning should allow {\em verifying
    set constraints}, thus the set domain \( \adom \) should provide an operator
  \( \adomprove: \adom \times \symsets \rightarrow \{ \true, \false \} \),
  which conservatively attempts to verify that a symbolic set constraint
  holds under some abstract states, \ie ensures that, for all \( \astate,
  L \), \( \gammadom( \adomassume( \astate, L ) ) = \true \) implies that
  \( \gammadom( \astate ) \subseteq \gamma( L ) \).

\item \emph{Generalizing set abstractions.}
  The analysis of loops is commonly based on the computation of abstract
  post-fixpoints~\cite{cc:popl:77}, thus \( \adom \) should provide
  sound over-approximation of concrete states join.
  In the logical point of view, this amounts to computing a common weakening
  for two abstract constraints.
  This is performed by an operator \( \adomjoin: \adom \times \adom
  \rightarrow \adom \) such that, for all \( \astate_0, \astate_1 \),
  \( \gammadom( \astate_0 ) \cup \gammadom( \astate_1 ) \subseteq
  \gammadom( \adomjoin( \astate_0, \astate_1 ) ) \).
  Widening operator \( \adomwiden \) should satisfy the same property
  and ensure termination of any sequence of abstract iterates.

\item \emph{Deciding entailment over set abstractions.}
  Finally, the operator \( \adomisle: \adom \times \adom \rightarrow \{ \true,
  \false \} \) conservatively decides implication among abstract set
  constraints (by ensuring that \( \adomisle( \astate_0, \astate_1 ) =
  \true \Longrightarrow \gammadom( \astate_0 ) \subseteq \gammadom(
  \astate_1 ) \)), and allows verifying the convergence of abstract
  iterates.
\end{asparaitem}
%ac: Should we include the rename operator here?  It seems critical to a lot of what we do, and while we consider it fairly trivial for domains, it is a non-trivial, expensive operation for SMT and solver domains.
