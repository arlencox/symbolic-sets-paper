\section{Logic and Set Abstraction}
\label{sec:logic-and-set-abstraction}
In this section, we define the signature of a set abstract domain, by the
concretization function and the abstract operations it should provide.

\subsection{Set Abstraction}
\label{s:3:1:abs}
In this paper, we use symbols $\setvw$, $\setvx$, $\setvy$, and $\setvz$
to represent set variables in $\setvars$.
We are interested in purely symbolic set relations, and do not make any
assumption on the type of the elements of the sets those variables stand
for.
We let \( \values \) denote the set of all these elements.
A concrete state is a function \( \state: \setvars \rightarrow
\partsof{\values} \).
We write \( \states \) for the set of such elements.

As usual in program analysis, {\em set abstract domain} is defined by a
set of {\em abstract elements} \( \adom \) which describe the family of
logical properties it can express and a concretization function
\( \gammadom: \adom \rightarrow \partsof{\states} \) that maps each
element of \( \adom \) into the set of concrete states that satisfy it.
\begin{example}[(Non-)Emptiness set domain]
  \label{ex:1:mt}
  A very basic example of such domain is the {\em (non-)emptiness} domain
  that comprises the following elements:
  \begin{compactitem}
  \item \( \bot \) which denotes the unsatisfiable abstract constraint
    (\ie, \( \gammadom( \bot ) = \emptyset ) \));
  \item the functions from \( \setvars \) into \( \{ [=\emptyset],
    [\not=\emptyset], \top \} \), and where a function \( \astate \) of
    that form is concretized into the set of states \( \state \) such that
    \( \astate( \setvx ) = [=\emptyset] \) (\resp, \( \astate( \setvx ) =
    [=\emptyset] \)) implies that \( \state( \setvx ) = \emptyset \)
    (\resp, \( \state( \setvx ) \not= \emptyset \))
  \end{compactitem}
\end{example}

\subsection{Operations over Set Abstractions}
\label{s:3:2:sign}
We now formalize the list of the main operations needed so that we can use
a set abstract \( \adom \) domain for either of the static analyses shown
in Section~\ref{s:2:over}.

\paragraph{Remarkable lattice elements.}
Static analyses typically start with an unconstrained state.
This is materialized by a \( \adomtop \in \adom \) element with full
concretization, \ie, \( \gammadom( \adomtop ) = \states \).
Similarly abstract element \( \adombot \in \adom \) should describe
the unsatisfiable abstract constraint (\ie, \( \gammadom( \adombot ) =
\emptyset \)).
In Example~\ref{ex:1:mt}, \( \adombot \) is \( \bot \) and \( \adomtop \)
is \( \lambda (x \in \setvars) \cdot \top \).
Moreover, a static analysis often has to determine if an abstract state
describes unsatisfiable constraints.
Thus, \( \adom \) should provide an operator \( \adomisbot: \adom
\rightarrow \{ \true, \false \} \) such that \( \adomisbot( \astate )
= \true \Longrightarrow \gammadom( \astate ) = \emptyset \).

\paragraph{Forgetting a set variable.}
When an update takes place, static analysis tools have to drop parts or
all of the facts they know about the entities modified by the assignment.
In the case of set abstract domain \( \adom \), this should be performed
by an operator \( \adomforget: \adom \times \setvars \rightarrow \adom \).
% forget soundness property

\paragraph{Assuming and verifying set constraints.}
As noted in Section~\ref{s:2:over}, an important part of the set reasoning
required in static analysis consists in moving set constraints (\eg, derived
from the unfolding of an inductive predicate with set constraints) into the
set domain, and in using the constraints tracked in the set domain in order
to discharge constraints (\eg, so as to enable the folding of an inductive
predicate).
We define the following language in order to express such constraints:
\begin{definition}[Symbolic Sets]
  \label{d:1:symsets}
  {\em Symbolic sets} are defined by the grammar:
  \begin{align*}
    L (\in \symsets) ::=
    & \ L \wedge L \
    | \ E \subseteq E \
    | \ \card{X} = 1 \
    | \ \top \
    | \ \bot
    & E ::=
    & \ \emptyset \ | \ X \ | \ \comp{E} \ | \ E \cup E
  \end{align*}
\end{definition}
The meaning of these contraints is straightforward, but we give a formal
definition in Figure~\ref{f:4:symsets} for clarity.
A model of a set expression $E$ is a concrete state $\state$ and a set
of concrete values $c$.
A model of a logical expression $L$ is a concrete state $\state$.
The concretization is given in terms of the models relationship.
\begin{figure}[t]
  \begin{align*}
    & \state, c \models \emptyset \textrm{ iff } c = \emptyset
    \qquad
    \state, c \models X \textrm{ iff } c = \state(X)
    \\
    & \state, c \models \comp{E}
    \textrm{ iff }
    \state, c' \models{E} \textrm{ and } \forall v \in \values.
    \; v \in c \leftrightarrow v \not\in c'
    \\
    & \state, c \models E_1 \cup E_2
    \textrm{ iff }
    \state, c_1 \models E_1 \textrm{ and } \state, c_2 \models E_2
    \\
    & \qquad \qquad \qquad \qquad \qquad  \textrm{ and }
    \forall v \in \values. \; v \in c \leftrightarrow v \in c_1 \vee
    v \in c_2
    \\
    & \state \models L_1 \wedge L_2
    \textrm{ iff }
    \state \models L_1 \textrm{ and } \state \models L_2 \qquad
    \state \models \top \qquad \state \not\models \bot
    \\
    & \state \models E_1 \subseteq E_2
    \textrm{ iff }
    \state, c_1 \models E_1 \textrm{ and } \state, c_2 \models E_2
    \textrm{ and } \forall v \in \values. \; v \in c_1 \rightarrow v \in c_2
    \\
    & \state \models \card{E} = 1
    \textrm{ iff }
    \state, c \models E \textrm{ and } \exists v \in \values. \; c = \{v\}
    \\
    & \gamma(L) = \{\ \sigma \ |\  \sigma \models L\ \}
  \end{align*}
  \caption{Symbolic sets}
  \label{f:4:symsets}
\end{figure}
We shall also use the following derived logical forms for simplicity.
%% xr: add membership
%% xr: consider removing equality (too trivial)
\[
\begin{array}{rclcrcl}
  E_1 \cap E_2 & \defeq & \comp{(\comp{E_1} \cup \comp{E_2})}
  & \qquad &
  E_1 = E_2 & \defeq & E_1 \subseteq E_2 \wedge E_2 \subseteq E_1
  \\
  E_1 \setminus E_2 &  \defeq & E_1 \cap \comp{E_2}
  & \qquad &
  E_1 \uplus E_2 & \defeq & E_1 \cup E_2 \textrm{ if } E_1 \cap E_2 = \emptyset
  \\
\end{array}
\]
Based, on this language, we require \( \adom \) to provide the following
abstract operators:
\begin{compactitem}
\item \( \adomassume: \adom \times \symsets \rightarrow \adom \), which
  conservatively represents a constraint into an abstract state, \ie,
  ensures that, for all \( \astate, L \), \( \gammadom( \astate ) \cap
  \gamma( L ) \subseteq \gammadom( \adomassume( \astate, L ) ) \);
\item \( \adomprove: \adom \times \symsets \rightarrow \{ \true, \false \} \),
  which conservatively attempts to verify that a symbolic set constraint
  holds under some abstract states, \ie, ensures that, for all \( \astate,
  L \), \( \gammadom( \adomassume( \astate, L ) ) = \true \) implies that
  \( \gammadom( \astate ) \subseteq \gamma( L ) \).
\end{compactitem}
% add some example

\paragraph{Lattice operations.}
Finally, the analysis of loops is commonly based on the computation of
abstract post-fixpoints~\cite{cc:popl:77}, thus \( \adom \) should provide
sound over-approximation of concrete states join.
In the logical point of view, this amounts to computing a common weakening
for two abstract constraints.
This is performed by an operator \( \adomjoin: \adom \times \adom
\rightarrow \adom \) such that, for all \( \astate_0, \astate_1 \),
\( \gammadom( \astate_0 ) \cup \gammadom( \astate_1 ) \subseteq
\gammadom( \adomjoin( \astate_0, \astate_1 ) ) \).
Widening operator \( \adomwiden \) should satisfy the same property
and ensure termination of any sequence of abstract iterates.

Last, operator \( \adomisle: \adom \times \adom \rightarrow \{ \true,
\false \} \) conservatively decides implication among abstract set
constraints (by ensuring that \( \adomisle( \astate_0, \astate_1 ) =
\true \Longrightarrow \gammadom( \astate_0 ) \subseteq \gammadom(
\astate_1 ) \)), and allows to verify the convergence of abstract
iterates.

%ac: Should we include the rename operator here?  It seems critical to a lot of what we do, and while we consider it fairly trivial for domains, it is a non-trivial, expensive operation for SMT and solver domains.