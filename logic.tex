\section{Logic and Set Abstraction}
\label{sec:logic-and-set-abstraction}
In this section, we define the signature of a set abstract domain, by the
concretization function and the abstract operations it should provide.

\subsection{Set Abstraction}
\label{s:3:1:abs}
In this paper, we use symbols $\setvw$, $\setvx$, $\setvy$, and $\setvz$
to represent set variables in $\setvars$.
We are interested in purely symbolic set relations, and do not make any
assumption on the type of the elements of the sets those variables stand
for.
We let \( \values \) denote the set of all these elements.
A concrete state is a function \( \state: \setvars \rightarrow
\partsof{\values} \).
We write \( \states \) for the set of such elements.

As usual in program analysis, {\em set abstract domain} is defined by a
set of {\em abstract elements} \( \adom \) which describe the family of
logical properties it can express and a concretization function
\( \gammadom: \adom \rightarrow \partsof{\states} \) that maps each
element of \( \adom \) into the set of concrete states that satisfy it.
\begin{example}[(Non-)Emptiness set domain]
  \label{ex:1:mt}
  A very basic example of such domain is the {\em (non-)emptiness} domain
  that comprises the following elements:
  \begin{compactitem}
  \item \( \bot \) which denotes the unsatisfiable abstract constraint
    (\ie, \( \gammadom( \bot ) = \emptyset ) \));
  \item the functions from \( \setvars \) into \( \{ [=\emptyset],
    [\not=\emptyset], \top \} \), and where a function \( \astate \) of
    that form is concretized into the set of states \( \state \) such that
    \( \astate( \setvx ) = [=\emptyset] \) (\resp, \( \astate( \setvx ) =
    [=\emptyset] \)) implies that \( \state( \setvx ) = \emptyset \)
    (\resp, \( \state( \setvx ) \not= \emptyset \))
  \end{compactitem}
\end{example}

\subsection{Operations over Set Abstractions}
\label{s:3:2:sign}
We now formalize the list of the main operations needed so that we can use
a set abstract \( \adom \) domain for either of the static analyses shown
in Section~\ref{s:2:over}.

\paragraph{Remarkable lattice elements.}
Static analyses typically start with an unconstrained state.
This is materialized by a \( \adomtop \in \adom \) element with full
concretization, \ie, \( \gammadom( \adomtop ) = \states \).
Similarly abstract element \( \adombot \in \adom \) should describe
the unsatisfiable abstract constraint (\ie, \( \gammadom( \adombot ) =
\emptyset \)).
In Example~\ref{ex:1:mt}, \( \adombot \) is \( \bot \) and \( \adomtop \)
is \( \lambda (x \in \setvars) \cdot \top \).
Moreover, a static analysis often has to determine if an abstract state
describes unsatisfiable constraints.
Thus, \( \adom \) should provide an operator \( \adomisbot: \adom
\rightarrow \{ \true, \false \} \) such that \( \adomisbot( \astate )
= \true \Longrightarrow \gammadom( \astate ) = \emptyset \).

\paragraph{Forgetting a set variable.}
When an update takes place, static analysis tools have to drop parts or
all of the facts they know about the entities modified by the assignment.
In the case of set abstract domain \( \adom \), this should be performed
by an operator \( \adomforget: \adom \times \setvars \rightarrow \adom \).
% forget soundness property

\paragraph{Assuming and verifying set constraints.}
As noted in Section~\ref{s:2:over}, an important part of the set reasoning
required in static analysis consists in moving set constraints (\eg, derived
from the unfolding of an inductive predicate with set constraints) into the
set domain, and in using the constraints tracked in the set domain in order
to discharge constraints (\eg, so as to enable the folding of an inductive
predicate).
We define the following language in order to express such constraints:
\begin{definition}[Symbolic Sets]
  \label{d:1:symsets}
  {\em Symbolic sets} are defined by the following grammar:
  \begin{align*}
    L (\in \symsets) ::=
    & \ L \wedge L \
    | \ E \subseteq E \
    | \ \card{X} = 1 \
    | \ \top \
    | \ \bot
    & E ::=
    & \ \emptyset \ | \ X \ | \ \comp{E} \ | \ E \cup E
  \end{align*}
\end{definition}
The meaning of these contraints is straightforward, but we give a formal
definition in Figure~\ref{f:4:symsets} for clarity.
A model of a set expression $E$ is a concrete state $\state$ and a set
of concrete values $c$.
A model of a logical expression $L$ is a concrete state $\state$.
The concretization is given in terms of the models relationship.
\begin{figure}[t]
  \begin{align*}
    & \state, c \models \emptyset \textrm{ iff } c = \emptyset
    \qquad
    \state, c \models X \textrm{ iff } c = \state(X)
    \\
    & \state, c \models \comp{E}
    \textrm{ iff }
    \state, c' \models{E} \textrm{ and } \forall v \in \values.
    \; v \in c \leftrightarrow v \not\in c'
    \\
    & \state, c \models E_1 \cup E_2
    \textrm{ iff }
    \state, c_1 \models E_1 \textrm{ and } \state, c_2 \models E_2
    \\
    & \qquad \qquad \qquad \qquad \qquad  \textrm{ and }
    \forall v \in \values. \; v \in c \leftrightarrow v \in c_1 \vee
    v \in c_2
    \\
    & \state \models L_1 \wedge L_2
    \textrm{ iff }
    \state \models L_1 \textrm{ and } \state \models L_2 \qquad
    \state \models \top \qquad \state \not\models \bot
    \\
    & \state \models E_1 \subseteq E_2
    \textrm{ iff }
    \state, c_1 \models E_1 \textrm{ and } \state, c_2 \models E_2
    \textrm{ and } \forall v \in \values. \; v \in c_1 \rightarrow v \in c_2
    \\
    & \state \models \card{E} = 1
    \textrm{ iff }
    \state, c \models E \textrm{ and } \exists v \in \values. \; c = \{v\}
    \\
    & \gamma(L) = \{\ \sigma \ |\  \sigma \models L\ \}
  \end{align*}
  \caption{Symbolic sets}
  \label{f:4:symsets}
\end{figure}
We shall also use the following derived logical forms for simplicity.
%% xr: add membership
%% xr: consider removing equality (too trivial)
\[
\begin{array}{rclcrcl}
  E_1 \cap E_2 & \defeq & \comp{(\comp{E_1} \cup \comp{E_2})}
  & \qquad &
  E_1 = E_2 & \defeq & E_1 \subseteq E_2 \wedge E_2 \subseteq E_1
  \\
  E_1 \setminus E_2 &  \defeq & E_1 \cap \comp{E_2}
  & \qquad &
  E_1 \uplus E_2 & \defeq & E_1 \cup E_2 \textrm{ if } E_1 \cap E_2 = \emptyset
  \\
\end{array}
\]
Based, on this language, we require \( \adom \) to provide the following
abstract operators:
\begin{compactitem}
\item \( \adomassume: \adom \times \symsets \rightarrow \adom \), which
  conservatively represents a constraint into an abstract state, \ie,
  ensures that, for all \( \astate, L \), \( \gammadom( \astate ) \sinter
  \gamma( L ) \subseteq \gammadom( \adomassume( \astate, L ) ) \);
\item \( \adomprove: \adom \times \symsets \rightarrow \{ \true, \false \} \),
  which conservatively attempts to verify that a symbolic set constraint
  holds under some abstract states, \ie, ensures that, for all \( \astate,
  L \), \( \gammadom( \adomassume( \astate, L ) ) = \true \) implies that
  \( \gammadom( \astate ) \subseteq \gamma( L ) \).
\end{compactitem}
% add some example

\paragraph{Lattice operations.}
% join, widening, le

\paragraph{Old version.}

A \emph{purely symbolic set abstraction} is an abstraction for sets that
places no requirements on the type or capabilities of the values stored
within the set.

A symbolic set consists of basic set combination operations with equality,
non-strict subset and simple cardinality.
While more general symbolic sets can reasoned about by extending the
cardinality (as in BAPA), the goal here is to perform invariant generation,
so the problem goes beyond that of deciding implication.
We must also approximate upper bound operations.
As a result, the logic is simplified in such a way that is simultaneously
useful and more amenable to computing upper bounds.



\subsection{Relationship with model checking}

Because set abstractions are closely related to Boolean algebras, it might be natural to wonder how set abstractions are related to hardware model checking.  There are existing fast algorithms for hardware model checking and the problem has been thoroughly studied.

The key differentiator between most abstract interpretation and model checking techniques is that model checking techniques are designed to be property directed.  A \emph{property directed} approach uses a known property that is trying to be proven as an input to invariant generation.  As a result, invariants that are generated are perhaps only suitable to prove that particular property (and weaker properties).  While abstract interpretation can be property directed, often it is not.  The purpose of abstract interpretation is to discover a superset of program behaviors and use those behaviors to prove many properties of programs.  Thus the goal of a property directed technique is different to that of most abstract interpreters.

Furthermore, the libraries that we describe in this paper are designed to be used in an externally defined abstract interpretation (Figure~\ref{fig:domain-usage}).  As a result, they are not designed to have full knowledge of the overall analysis.  They are instead designed to reason about set constraints describing a single abstract state.  The understanding of the analysis is relegated to the external entity that performs the abstract interpretation.  In other words, a set abstraction is to an abstract interpreter what a sat solver is to a hardware model checker.  The set abstractions are different from SAT solvers because they are designed to solve the kinds of problems that arise in abstract interpretation.

\begin{figure}[tb]
    \centering
    \begin{tikzpicture}[
    every on chain/.append style={join},
    every join/.style={->}
    ]
    \matrix (m) [matrix of nodes, column sep=5mm, nodes={}]
    {
        Abstract Interpreter & Memcad & Set Abstraction \\
        Abstract Interpreter & HOO & Set Abstraction \\
        Abstract Interpreter &     & Set Abstraction \\
    };
    { [start chain]
        \chainin (m-1-1);
        \chainin (m-1-2);
        \chainin (m-1-3);
    }
    { [start chain]
        \chainin (m-2-1);
        \chainin (m-2-2);
        \chainin (m-2-3);
    }
    { [start chain]
        \chainin (m-3-1);
        \chainin (m-3-3);
    }
    \end{tikzpicture}
    \caption{A set abstraction is used as part a larger analysis.  The abstract interpreter delegates some tasks to abstract domains.  Here, we consider three cases (1) Memcad is an abstract domain for C heaps that delegates some problems to a set abstraction; (2) HOO is an abstract domain for JavaScript heaps that delegates many problems to a set abstraction; and (3) The set abstraction is used directly}
    \label{fig:domain-usage}
\end{figure}

\paragraph{Key challenges:}  As opposed to model checkers, there are three additional challenges that a symbolic set abstraction faces. (1) Forward analysis of assignment introduces existential quantifiers.  They must be eliminated or otherwise handled by the abstraction.  The nesting of these quantifiers can be very deep. (2) The constraints are equality heavy and disjointness heavy.  Many constraints are simple equality ($X = Y$) or disjointness ($X \cap Y = \emptyset$).  (3) There are often many variables with few constraints on them.  Many variables may be introduced without being significantly constrained.

\paragraph{Key reliefs:}  As opposed to model checkers, symbolic set abstraction is easier in () ways. (1) There are often fewer variables.  Since not all variables are set variables, the total number of variables is significantly lower.  (2) Relationships between variables are not as complex.  Because the transition relation does not have to be included in the Boolean formulas, the set abstraction is only responsible for relationships that form between set variables.  These can be much simpler than what the full transition relation would dictate.  (3) Variables are often clustered into small groups that are tightly inter-related.  This means that packing and slicing techniques can be extremely beneficial.