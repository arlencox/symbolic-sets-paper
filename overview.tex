\section{Overview}
\label{s:2:over}
In this section, we present two static analyses that make use of set
reasoning in order to compute high level semantic properties of
programs.
These analyses rely on abstract interpretation~\cite{cc:popl:77} and
on an abstraction of program states, that describes data structures
and their contents.
An abstract domain defines a set of predicates that an analysis may
use, as well as operators to over-approximate the effect of program
behaviors on these predicates, and their implementation.

\paragraph{Inferrence of properties of open objects.}
Dynamic programming languages such as JavaScript feature {\em open objects}
that support dynamic addition and deletion of attributes and iteration over
them.
The analysis presented in~\cite{hoo:14:sas} computes relations
among objects, so as to verify programs as the piece of code of
Figure~\ref{f:1:intro}.
To achieve this, it infers relations between the sets of attributes
of distinct objects.
Objects have an unbounded number of attributes, thus the analysis requires
some abstraction over the attributes and their contents.
\newcommand{\varsrc}{\cvar{src}}
\newcommand{\vardst}{\cvar{dst}}
\begin{figure}[t]
  \newcommand{\picscale}{0.82}
  \begin{center}
    \tikzpics{\picscale}{hoo-inv}
  \end{center}
  \caption{Open objects and their abstraction}
  \label{f:2:hoo}
\end{figure}
Figure~\ref{f:2:hoo} represents a very simplified state, at the loop
head and after two iteration (thus two fields were copied).
We focus on the set of attributes of each object, and ignore their
contents (which could be described using similar techniques).
To abstract precisely the relations between the attributes of both objects
(\ie, in this case, copied attributes are common to both objects), we need
to describe the fields of each object as the union of a series of sets of
attributes, and to express relations among these sets.
The first purpose of the set abstract domain is to represent such set
relations.
The right hand side of Figure~\ref{f:2:hoo} depicts such an abstract
state, where \( X_n, X_r, X_v \) stand for sets of attributes, which are
made explicit, in the case of the left hand side concrete state.

Moreover, to infer these invariants, the analysis needs to reason
both about object structures and about attributes sets:
\begin{asparaitem}
\item Initially, no set relation should be assumed, and the fields of
  each object should be associated to a set of attributes;
\item When the analysis enters the body of the loop, it needs to
  {\em single out} attribute \( \mathtt{x} \), thus, to replace set
  \( X_v \) by \( X_v \suplus \{ \mathtt{x} \} \) (which produces the
  equalities of Figure~\ref{f:2:hoo});
\item When it exits the loop, the analysis should {\em generalize}
  both the object and set constraints abstractions, which requires
  to {\em eliminate} singleton \( \{ \mathtt{x} \} \) from the
  equations (it is visible only in the loop body) and to synthesize
  a new, more general collection of constraints.
\end{asparaitem}
To allow these steps, the set abstraction should provide basic operations
over set predicates, including (1) the addition of a set constraint, (2)
the proving of a set constraint, (3) the removal of a set variable and
(4) the generalization of two set abstract states.

\paragraph{Shape analysis in presence of unstructured sharing.}
The shape analysis for data-structures with unbounded sharing presented
in~\cite{memcad:15:sas} relies on separation logic~\cite{r:lics:02} to
describe memory states, and on inductive definitions to summarize
unbounded structures such as lists.
Unstructured sharing is very challenging as it cannot be described using
conventional inductive definitions.
Figure~\ref{f:3a:memcad:conc} displays the representation of a three nodes
graph using an adjacency list data-structure.
To summarize such a structure using inductive predicates in separation
logic, \cite{memcad:15:sas} proposes to augment the list inductive
predicates with set information, which express where edges may point to.
This representation is shown in Figure~\ref{f:3b:memcad:sum} in a form
where the first node is kept materialized.
It both asserts that edges of that node as well as edges from other nodes
point to the address of a valid node, namely an element of \( \{ \naddr{0}
\} \suplus \mcE \).
\begin{figure}[t]
  \newcommand{\picscale}{0.9}
  \begin{center}
    \subfigure[A concrete state]{ \label{f:3a:memcad:conc}
      \tikzpics{\picscale}{memcad-conc}
    }
    \quad
    \subfigure[Summarization principle]{ \label{f:3b:memcad:sum}
      \tikzpics{\picscale}{memcad-abs}
    }
  \end{center}
  \caption{Summarization of an adjacency list-based graph representation}
  \label{f:3:memcad}
\end{figure}
The analysis of~\cite{memcad:15:sas} introduces a summary predicate
\( \textbf{graph}( \naddr{0}, \mcN ) \) where \( \naddr{0} \) is the
address of the first node and \( \mcN \) the set of all node addresses.
This predicate is defined by induction over the ``backbone'' of the
structure, and fully takes into account the property that all
\( \fldnext \) edges point to a valid node address in \( \naddr{0} \).
Henceforth, abstract states comprise both a {\em memory} part (which
consists of a formlula in separation logic with inductive predicates)
and a {\em set abstraction}.

To compute such summaries, the analysis needs to perform similar
operations as the analysis for open objects, in order to add set
constraints to the set abstract state, to prove set constraints,
to remove set variables and to generalize abstract states.

% \paragraph{Need for set reasoning.}
% Both analyses consist of a composite abstract domain, that comprises a
% {\em structural abstract domain} describing the shape of structures, and
% of a {\em set abstract domain} expressing relations among set and value
% (pointers, field names, contents...) entities.
