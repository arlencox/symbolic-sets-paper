\section{Overview}
\label{s:2:over}
In this section, we present two static analyses that make use of set
reasoning, in order to compute high level semantic properties of
programs.
These analyses rely on abstract interpretation~\cite{cc:popl:77} and
on an abstraction of program states, that describes data structures
and their contents.
An abstract domain defines a set of predicates that an analysis may
use, as well as operators to over-approximate the effect of program
behaviors on these predicates, and their implementation.

\paragraph{Inferrence of properties of open objects.}
%   . main features of open objects
%   . static analysis task: infer relation between object sets of fields
%   . abstraction of structures
%   . role of the set abstraction
%   . nature of the invariants that are inferred
\begin{figure}[t]
  \subfigure[A concrete state]{
    % (a) a concrete state
  }
  \quad
  \subfigure[Corresponding abstract state]{
    % (b) its abstraction
  }
  \caption{Open objects and their abstraction}
  \label{f:1:hoo}
\end{figure}
\cite{hoo:14:sas}

\paragraph{Shape analysis in presence of unstructured sharing.}
% - paragraph on MemCAD
%   . sharing properties, and inductive definitions parametered with sets
%   . static analysis tasks: infer summaries, with appropriate set parameters
%   . abstract domain structure
%   . role of the set abstraction
\begin{figure}[t]
  \subfigure[A concrete state]{
    % (a) a concrete state
  }
  \quad
  \subfigure[Corresponding abstract state]{
    % (b) its abstraction
  }
  \caption{Shape abstraction and sharing}
  \label{f:2:memcad}
\end{figure}
\cite{memcad:15:sas}

\paragraph{Need for set reasoning.}
%   . a "master domain" controls the shape of properties
%   . set abstraction not specific, simply represents the facts sent by
%     the "master domain".
