%-----------------------------------------%
% Colors                                  %
%-----------------------------------------%
\newrgbcolor{mblue}{        0.1   0.1   0.5 }
\newrgbcolor{mbrown}{       0.57  0.4   0.3 }
\newrgbcolor{mcyan}{        0.1   0.1   0.5 }
\newrgbcolor{mgreen}{       0.2   0.5   0.2 }
\newrgbcolor{mmidgray}{     0.65  0.65  0.65}
\newrgbcolor{morange}{      0.8   0.45  0.45}
\newrgbcolor{mpurple}{      0.5   0.1   0.5 }
\newrgbcolor{mred}{         1     0     0   }
\newrgbcolor{mhigray}{      0.21  0.21  0.21}
\newrgbcolor{mgray}{        0.5   0.5   0.5 }
\newrgbcolor{mmidgreen}{    0.3   0.8   0.4 }
\newrgbcolor{mlightgray}{   0.85  0.85  0.85}
\newrgbcolor{mlightblue}{   0.7   0.85  1.00}
\newrgbcolor{mlightgreen}{  0.7   1.00  0.85}
\newrgbcolor{mlightpurple}{ 1.00  0.7   1.00}
\newrgbcolor{mlightred} {   1.00  0.7   0.7 }
\newrgbcolor{mlightyellow}{ 1.00  1.00  0.6 }
\newrgbcolor{mulightgray}{  0.98  0.98  0.98}
\newrgbcolor{mulightblue}{  0.9   0.95  1.00}
\newrgbcolor{mulightgreen}{ 0.9   1.00  0.95}
\newrgbcolor{mulightpurple}{1.00  0.9   1.00}
\newrgbcolor{mulightred} {  1.00  0.85  0.85}
\newrgbcolor{mulightyellow}{1.00  1.00  0.87}
\newrgbcolor{mvlightblue}{  0.8   0.9   1.00}
\newrgbcolor{mvlightgreen}{ 0.8   1.00  0.9 }
\newrgbcolor{mvlightpurple}{1.00  0.8   1.00}
\newrgbcolor{mvlightred} {  1.00  0.85  0.85}
\newrgbcolor{mvlightyellow}{1.00  1.00  0.73}
\newrgbcolor{myellow}{      1.00  1.00  0.30}
\newrgbcolor{mwhite}{       0.99  0.99  0.99}

%-----------------------------------------%
% Typing and abbreviations                %
%-----------------------------------------%
% Standard Abbrevs
\newcommand{\eg}{e.g.\xspace}
\newcommand{\ie}{i.e.\xspace}
\newcommand{\resp}{resp.\xspace}
% Removing
\newcommand{\commentout}[1]{}
% Project names
\newcommand{\memcad}{{MemCAD}\xspace}

%-----------------------------------------%
% Math standard stuffs                    %
%-----------------------------------------%
% Abstraction
\newcommand{\abs}[1]{{#1}^{\sharp}}
% Booleans
\newcommand{\true}{\mathbf{true}}
\newcommand{\false}{\mathbf{false}}
% Definitions
\newcommand{\isdef}{::=}
% Logical relations
\newcommand{\logor}{\mathrel{\vee}}
\newcommand{\logand}{\mathrel{\wedge}}
\newcommand{\suplus}{\mathrel{\uplus}}
% Powerset
\newcommand{\partsof}[1]{\mcP(#1)}
% Semantics
\newcommand{\sem}[1]{\llbracket #1 \rrbracket}
\newcommand{\asem}[1]{\abs{\llbracket #1 \rrbracket}}

%-----------------------------------------%
% Inserting Tikz pictures                 %
%-----------------------------------------%
% Including pictures
\newcommand{\tikzpics}[2]{\scalebox{#1}{\input{tkz-#2.tex}}} % with scale
\newcommand{\tikzpic}[1]{\tikzpics{1}{#1}}                   % without scale
% Putting a formula
\newcommand{\tkzputform}[2]{\node[] at (#1){\ensuremath{#2}};}
\newcommand{\tkzputrghform}[2]{
  \node[label distance=-10pt,inner sep=0pt,outer sep=0pt,
    minimum height=0pt,minimum width=0pt,
    outer xsep=0pt,outer ysep=0pt,inner xsep=0pt,inner ysep=0pt,
    label=right:\ensuremath{#2}] at (#1){};}
\newcommand{\tkzputlftform}[2]{
  \node[label distance=-10pt,inner sep=0pt,outer sep=0pt,
    minimum height=0pt,minimum width=0pt,
    outer xsep=0pt,outer ysep=0pt,inner xsep=0pt,inner ysep=0pt,
    label=left:\ensuremath{#2}] at (#1){};}
% Putting a piece of text
\newcommand{\tkzputtext}[2]{\node[] at (#1){#2};}
\newcommand{\tkzputrghtext}[2]{
  \node[label distance=-10pt,inner sep=0pt,outer sep=0pt,
    minimum height=0pt,minimum width=0pt,
    outer xsep=0pt,outer ysep=0pt,inner xsep=0pt,inner ysep=0pt,
    label=right:#2] at (#1){};}
\newcommand{\tkzputlfttext}[2]{
  \node[label distance=-10pt,inner sep=0pt,outer sep=0pt,
    minimum height=0pt,minimum width=0pt,
    outer xsep=0pt,outer ysep=0pt,inner xsep=0pt,inner ysep=0pt,
    label=left:#2] at (#1){};}
% Pointers
\newcommand{\tkptrdot}[1]{\draw (#1) [fill=black] circle (0.05);}

%-----------------------------------------%
% Programs                                %
%-----------------------------------------%
% Variables
\newcommand{\ttvar}[1]{\mathtt{#1}}
\newcommand{\cvar}[1]{\ensuremath{\texttt{#1}}}
\newcommand{\varc}{\ttvar{c}}
\newcommand{\vard}{\ttvar{d}}
\newcommand{\varg}{\ttvar{g}}
\newcommand{\varh}{\ttvar{h}}
\newcommand{\vari}{\ttvar{i}}
\newcommand{\varl}{\ttvar{l}}
\newcommand{\varm}{\ttvar{m}}
\newcommand{\varn}{\ttvar{n}}
\newcommand{\varp}{\ttvar{p}}
\newcommand{\varr}{\ttvar{r}}
\newcommand{\vars}{\ttvar{s}}
\newcommand{\vart}{\ttvar{t}}
\newcommand{\varu}{\ttvar{u}}
\newcommand{\varv}{\ttvar{v}}
\newcommand{\varx}{\ttvar{x}}
\newcommand{\vary}{\ttvar{y}}
% Pointers
\newcommand{\nullptr}{\textbf{0x0}}

%-----------------------------------------%
% single abstract state
\renewcommand{\abstract}{\Sigma}
% set of all abstract states
\newcommand{\abstracts}{\textsc{AbsStates}}
% cardinality of a set
\newcommand{\card}[1]{|{#1}|}
% complement of a set
\newcommand{\comp}[1]{{#1}^{\mathrm{c}}}
% individual set variables: W, X, Y, or Z
% set of all set variables
\newcommand{\setvars}{\textsc{SetVars}}
% concrete state
\newcommand{\state}{\sigma}
% set of all concrete states
\newcommand{\states}{\textsc{States}}
% set of all values
\newcommand{\values}{\textsc{Vals}}
% concrete set c
% syntax of a logical constraint L
% syntax of a set expression E
% power set of a set
\newcommand{\powerset}[1]{\mathcal{P}\left({#1}\right)}


\newcommand{\defeq}{\stackrel{\textrm{\tiny def}}{=}}