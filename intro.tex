\section{Introduction}
\label{s:1:intro}

% Don't forget to add references!!!

It is well understood that the verification of program properties that
involve data structures is a challenging problem~\cite{jahob:thesis:07,compass:popl:11,fixbag:cav:11,celia:vmcai:12,ab:ecoop:13,hoo:14:sas,memcad:15:sas}.
There are a multitude of reasons for this, but one key reason is that
if a data structure is unbounded, there is a potentially unbounded number of constraints on its elements.  Since these constraints often affect important properties such as memory safety~\cite{memcad:15:sas}, functional correctness~\cite{fixbag:cav:11}, or basic program behavior~\cite{hoo:14:sas}, it is vital to develop techniques for efficiently reasoning about relationships between unbounded numbers of elements.

Here we use set constraints to reason about unbounded collections of elements.  Set constraints can be used to dynamically partition data structures, correlate collections of elements with one another, or determine analysis case splits.  They are useful for representing data and pointers relationships in structures such as maps, graphs, lists, sets, and arrays.  They can be combined with other techniques such as separation logic~\cite{hoo:14:sas,memcad:15:sas} and numerical analyses~\cite{quicr:cav:14} to enhance those analyses.

%% xr: need to adjust overview, there is some repetition now
For example, consider the program in Figure~\ref{fig:intro-example}
that copies one map on top of another.
Within the loop, there is a complex relationship between the sets of
keys of \ttvar{src} and \ttvar{dst}.
At the specified point, the keys of \ttvar{src} can be partitioned into
three parts.
The keys $X_v$ already visited by the loop, the element currently being
visited by the loop $\{\varx\}$, and the keys $X_n$ not visited by
the loop.
The keys of \ttvar{dst} can be partitioned into those originally in
\ttvar{dst} that have not been overwritten, and those $X_v$ that have
been overwritten or added from \ttvar{src}.
This set reasoning allows precise symbolic tracking of the provenance
of map partitions.

\begin{figure}[tb]
  \newbox\exprogbox
  \begin{lrbox}{\exprogbox}
    \begin{minipage}[t][1cm][b]{0.4\textwidth}
      \begin{lstlisting}[language=python]
def extend(dst, src):
  for x in src:
    # invariant point
    dst[x] = src[x]
      \end{lstlisting}
    \end{minipage}
  \end{lrbox}
  \newbox\exproginv
  \begin{lrbox}{\exproginv}
    \begin{minipage}[t][0.9cm][b]{0.4\textwidth}
      \begin{align*}
        \exists X_v, X_n. \; keys(\ttvar{src})
        & = X_v \suplus \{\varx\} \suplus X_n
        \\
        {} \logand keys(\ttvar{dst})
        & = (keys(\ttvar{dst})_0 \setminus X_v) \suplus X_v
      \end{align*}
    \end{minipage}
  \end{lrbox}
  \centering
  \subfigure[Function to copy all keys and values from map \ttvar{src}
  onto map \ttvar{dst}]{\usebox\exprogbox}
  \quad
  \subfigure[Set abstraction at \texttt{\textit{invariant point}}]{%
    \usebox\exproginv}
  \caption{Set constraints can relate portions of data structures}
  \label{fig:intro-example}
\end{figure}

This paper focuses on the subset of set logic that is \emph{nearly
  Boolean algebraic}---a Boolean algebra over set variables with
singleton sets.
We find that this subset is sufficiently large to be useful and we
believe that it serves as a good starting point for extensions to the
logic, such as reasoning about explicit set contents or more precise
cardinality.

However, even with the nearly Boolean algebraic restriction, naive
implementations of invariant generation can easily be intractably
slow and uselessly imprecise.
%% xr: this part seems unconvincing to me
%%     maybe the issue is that "combining...etc" seems a bit abstract here
%%     [we have a good idea of how we do it, but I do not think the reader
%%      will]
The primary source of performance problems is the Boolean algebra.
By combining unions, intersections, complements, conjunctions,
disjunctions, and quantification, the representation of the Boolean
algebra \emph{often} grows exponentially large or takes exponential
time to query.
%% xr: what is meant by "simple attempts"
Simple attempts to limit this exponential behavior lead to unpredictable
imprecision.

% AC: seems like this could be tightened up and combined with the next paragraph
In this paper we aim to address the performance and precision challenges
by defining a variety of interchangeable abstractions and abstraction
combinators that can be selected for a particular application.
This includes abstractions based on binary decision diagrams,
satisfiability modulo theories, and linear set constraints.
Additionally, it includes performance enhancing combinators for tracking
singleton sets, handling equality, and doing dynamic variable packing.
These abstractions and combinators exist within a general framework so
that additional abstractions and combinators can be easily added.
This framework is available within the QUICr library, which is now being
used by two research analyzers.

In this paper we make the following contributions.
\begin{compactitem}
\item We define a common language for symbolic set constraints.
  %% xr: not very specific; I suggest to add "adapted for static analysis"
  %%     or "that describes static analyzers set reasoning needs"
  This constraint language is used as the framework for all set
  abstractions (Section~\ref{sec:logic-and-set-abstraction}).
\item We construct scalable abstractions for set relationships
  targeted at real-world problems in data structure verification.
  %% xr: our construction allows a lot of flexibility; we should
  %%     probably highlight this (constructed abstractions can be
  %%     adapted both in cost and precision to the needs)
  These abstractions use specialized data structures to represent set
  constraints uniformly and efficiently (Section~\ref{sec:constructed}).
\item We adapt existing solvers for Boolean algebras to the set abstraction
  framework.
  %% xr: equivalently => with what ?
  We show that this is equivalently expressive (Section~\ref{sec:solvers}).
\item We compare and contrast the constructed abstractions, solver
  abstractions, with and without various abstraction combinators on a
  variety of real-world benchmarks.
  %% xr: "can offer" make it seem incidental.
  We find that constructed abstractions can offer significantly better
  performance than existing solvers without any loss of precision
  (Section~\ref{sec:evaluation}).
\end{compactitem}