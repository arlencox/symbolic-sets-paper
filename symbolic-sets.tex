\documentclass{llncs}
\usepackage{etex}
\usepackage{tikz}
\usetikzlibrary{matrix,chains,scopes}
\usetikzlibrary{arrows.meta}
\usetikzlibrary{patterns}
\usepackage{booktabs}
\usepackage{amsmath,amssymb,amsfonts,amstext}
\usepackage{color}
\usepackage{graphicx}
\usepackage{microtype}
\usepackage{paralist}
\usepackage{pst-node}
\RequirePackage{stmaryrd}
\RequirePackage{wrapfig}
\usepackage{xspace}
\usepackage{listings}
\usepackage{subfigure}
\usepackage{braket}
\usepackage{siunitx}
\usepackage{tabularx}
\usepackage[subtle]{savetrees}
%\usepackage{times}
\usepackage[numbers,sort&compress]{natbib}
\renewcommand{\ttdefault}{pcr}

\makeatletter
\renewcommand\bibsection%
{
    \section*{\refname
        \@mkboth{\MakeUppercase{\refname}}{\MakeUppercase{\refname}}}
}
\makeatother

\lstdefinestyle{colored}{
    belowskip=0pt,
    xleftmargin=\parindent,
    showstringspaces=true,
    basicstyle=\footnotesize\ttfamily,
    keywordstyle=\bfseries\color{green!40!black},
    commentstyle=\itshape\color{purple!40!black},
%    identifierstyle=\color{blue},
    stringstyle=\color{orange}
}

\lstset{basicstyle=\ttfamily,style=colored}

%----------------------------------------------------------
%-% Definition of fonts for Maths

%----------------------------------------------------------
%-% Math Bold fonts
\usepackage[bbgreekl]{mathbbol}
\newcommand{\mathboldcommand}[1]{\mathbb{#1}}
\newcommand{\bbA}{\mathboldcommand{A}}
\newcommand{\bbB}{\mathboldcommand{B}}
\newcommand{\bbC}{\mathboldcommand{C}}
\newcommand{\bbD}{\mathboldcommand{D}}
\newcommand{\bbE}{\mathboldcommand{E}}
\newcommand{\bbF}{\mathboldcommand{F}}
\newcommand{\bbG}{\mathboldcommand{G}}
\newcommand{\bbH}{\mathboldcommand{H}}
\newcommand{\bbI}{\mathboldcommand{I}}
\newcommand{\bbJ}{\mathboldcommand{J}}
\newcommand{\bbK}{\mathboldcommand{K}}
\newcommand{\bbL}{\mathboldcommand{L}}
\newcommand{\bbM}{\mathboldcommand{M}}
\newcommand{\bbN}{\mathboldcommand{N}}
\newcommand{\bbO}{\mathboldcommand{O}}
\newcommand{\bbP}{\mathboldcommand{P}}
\newcommand{\bbQ}{\mathboldcommand{Q}}
\newcommand{\bbR}{\mathboldcommand{R}}
\newcommand{\bbS}{\mathboldcommand{S}}
\newcommand{\bbT}{\mathboldcommand{T}}
\newcommand{\bbU}{\mathboldcommand{U}}
\newcommand{\bbV}{\mathboldcommand{V}}
\newcommand{\bbW}{\mathboldcommand{W}}
\newcommand{\bbX}{\mathboldcommand{X}}
\newcommand{\bbY}{\mathboldcommand{Y}}
\newcommand{\bbZ}{\mathboldcommand{Z}}
\newcommand{\bba}{\mathboldcommand{a}}
\newcommand{\bbb}{\mathboldcommand{b}}
\newcommand{\bbc}{\mathboldcommand{c}}
\newcommand{\bbd}{\mathboldcommand{d}}
\newcommand{\bbe}{\mathboldcommand{e}}
\newcommand{\bbf}{\mathboldcommand{f}}
\newcommand{\bbg}{\mathboldcommand{g}}
\newcommand{\bbh}{\mathboldcommand{h}}
\newcommand{\bbi}{\mathboldcommand{i}}
\newcommand{\bbj}{\mathboldcommand{j}}
\newcommand{\bbk}{\mathboldcommand{k}}
\newcommand{\bbl}{\mathboldcommand{l}}
\newcommand{\bbm}{\mathboldcommand{m}}
\newcommand{\bbn}{\mathboldcommand{n}}
\newcommand{\bbo}{\mathboldcommand{o}}
\newcommand{\bbp}{\mathboldcommand{p}}
\newcommand{\bbq}{\mathboldcommand{q}}
\newcommand{\bbr}{\mathboldcommand{r}}
\newcommand{\bbs}{\mathboldcommand{s}}
\newcommand{\bbt}{\mathboldcommand{t}}
\newcommand{\bbu}{\mathboldcommand{u}}
\newcommand{\bbv}{\mathboldcommand{v}}
\newcommand{\bbw}{\mathboldcommand{w}}
\newcommand{\bbx}{\mathboldcommand{x}}
\newcommand{\bby}{\mathboldcommand{y}}
\newcommand{\bbz}{\mathboldcommand{z}}
\newcommand{\bbga}{\mathboldcommand{\bbalpha}}
\newcommand{\bbgb}{\mathboldcommand{\bbbeta}}
\newcommand{\bbgc}{\mathboldcommand{\bbgamma}}
\newcommand{\bbgd}{\mathboldcommand{\bbdelta}}
\newcommand{\bbge}{\mathboldcommand{\bbepsilon}}
\newcommand{\bbgi}{\mathboldcommand{\bbiota}}
\newcommand{\bbgt}{\mathboldcommand{\bbtau}}
\newcommand{\bblbracket}{\mathboldcommand{\Lbrack}}
\newcommand{\bbrbracket}{\mathboldcommand{\Rbrack}}
\newcommand{\bblparen}{\llparenthesis}
\newcommand{\bbrparen}{\rrparenthesis}
\newcommand{\bblangle}{\mathboldcommand{\Langle}}
\newcommand{\bbrangle}{\mathboldcommand{\Rangle}}
\newcommand{\bblbrace}{\mathboldcommand{\{}} % KO
\newcommand{\bbrbrace}{\mathboldcommand{\}}} % KO
%----------------------------------------------------------

%----------------------------------------------------------
%-% Math caligraphic characters
\usepackage{euscript}
\newcommand{\mathcalcommand}[1]{\mathcal{#1}}
\newcommand{\mcA}{\mathcalcommand{A}}
\newcommand{\mcB}{\mathcalcommand{B}}
\newcommand{\mcC}{\mathcalcommand{C}}
\newcommand{\mcD}{\mathcalcommand{D}}
\newcommand{\mcE}{\mathcalcommand{E}}
\newcommand{\mcF}{\mathcalcommand{F}}
\newcommand{\mcG}{\mathcalcommand{G}}
\newcommand{\mcH}{\mathcalcommand{H}}
\newcommand{\mcI}{\mathcalcommand{I}}
\newcommand{\mcJ}{\mathcalcommand{J}}
\newcommand{\mcK}{\mathcalcommand{K}}
\newcommand{\mcL}{\mathcalcommand{L}}
\newcommand{\mcM}{\mathcalcommand{M}}
\newcommand{\mcN}{\mathcalcommand{N}}
\newcommand{\mcO}{\mathcalcommand{O}}
\newcommand{\mcP}{\mathcalcommand{P}}
\newcommand{\mcQ}{\mathcalcommand{Q}}
\newcommand{\mcR}{\mathcalcommand{R}}
\newcommand{\mcS}{\mathcalcommand{S}}
\newcommand{\mcT}{\mathcalcommand{T}}
\newcommand{\mcU}{\mathcalcommand{U}}
\newcommand{\mcV}{\mathcalcommand{V}}
\newcommand{\mcW}{\mathcalcommand{W}}
\newcommand{\mcX}{\mathcalcommand{X}}
\newcommand{\mcY}{\mathcalcommand{Y}}
\newcommand{\mcZ}{\mathcalcommand{Z}}
%----------------------------------------------------------

%----------------------------------------------------------
%-% Math Zaph Chancery characters
\DeclareMathAlphabet{\mathpzc}{T1}{pzc}{m}{it}
\newcommand{\zaphcommand}[1]{\mathpzc{#1}}
\newcommand{\bigzaphcommand}[1]{
  \text{\fontsize{14}{10}\usefont{T1}{pzc}{m}{it}{#1}}}
%\newcommand{\zaphcommand}[1]{
%  \text{\fontsize{11}{10}\usefont{T1}{pzc}{m}{it}{#1}}}
\newcommand{\zcA}{\zaphcommand{A}}
\newcommand{\zcB}{\zaphcommand{B}}
\newcommand{\zcC}{\zaphcommand{C}}
\newcommand{\zcD}{\zaphcommand{D}}
\newcommand{\zcE}{\zaphcommand{E}}
\newcommand{\zcF}{\zaphcommand{F}}
\newcommand{\zcG}{\zaphcommand{G}}
\newcommand{\zcH}{\zaphcommand{H}}
\newcommand{\zcI}{\zaphcommand{I}}
\newcommand{\zcJ}{\zaphcommand{J}}
\newcommand{\zcK}{\zaphcommand{K}}
\newcommand{\zcL}{\zaphcommand{L}}
\newcommand{\zcM}{\zaphcommand{M}}
\newcommand{\zcN}{\zaphcommand{N}}
\newcommand{\zcO}{\zaphcommand{O}}
\newcommand{\zcP}{\zaphcommand{P}}
\newcommand{\zcQ}{\zaphcommand{Q}}
\newcommand{\zcR}{\zaphcommand{R}}
\newcommand{\zcS}{\zaphcommand{S}}
\newcommand{\zcT}{\zaphcommand{T}}
\newcommand{\zcU}{\zaphcommand{U}}
\newcommand{\zcV}{\zaphcommand{V}}
\newcommand{\zcW}{\zaphcommand{W}}
\newcommand{\zcX}{\zaphcommand{X}}
\newcommand{\zcY}{\zaphcommand{Y}}
\newcommand{\zcZ}{\zaphcommand{Z}}
\newcommand{\zca}{\zaphcommand{a}}
\newcommand{\zcb}{\zaphcommand{b}}
\newcommand{\zcc}{\zaphcommand{c}}
\newcommand{\zcd}{\zaphcommand{d}}
\newcommand{\zce}{\zaphcommand{e}}
\newcommand{\zcf}{\zaphcommand{f}}
\newcommand{\zcg}{\zaphcommand{g}}
\newcommand{\zch}{\zaphcommand{h}}
\newcommand{\zci}{\zaphcommand{i}}
\newcommand{\zcj}{\zaphcommand{j}}
\newcommand{\zck}{\zaphcommand{k}}
\newcommand{\zcl}{\zaphcommand{l}}
\newcommand{\zcm}{\zaphcommand{m}}
\newcommand{\zcn}{\zaphcommand{n}}
\newcommand{\zco}{\zaphcommand{o}}
\newcommand{\zcp}{\zaphcommand{p}}
\newcommand{\zcq}{\zaphcommand{q}}
\newcommand{\zcr}{\zaphcommand{r}}
\newcommand{\zcs}{s}
\newcommand{\zct}{\zaphcommand{t}}
\newcommand{\zcu}{\zaphcommand{u}}
\newcommand{\zcv}{\zaphcommand{v}}
\newcommand{\zcw}{\zaphcommand{w}}
\newcommand{\zcx}{\zaphcommand{x}}
\newcommand{\zcy}{\zaphcommand{y}}
\newcommand{\zcz}{\zaphcommand{z}}
\newcommand{\bzcd}{\bigzaphcommand{d}}
%----------------------------------------------------------

%----------------------------------------------------------
%-% Math Fraktur fonts
%\usepackage{eufrak}
\newcommand{\mathfrakcommand}[1]{\mathfrak{#1}}
\newcommand{\fkA}{\mathfrakcommand{A}}
\newcommand{\fkB}{\mathfrakcommand{B}}
\newcommand{\fkC}{\mathfrakcommand{C}}
\newcommand{\fkD}{\mathfrakcommand{D}}
\newcommand{\fkE}{\mathfrakcommand{E}}
\newcommand{\fkF}{\mathfrakcommand{F}}
\newcommand{\fkG}{\mathfrakcommand{G}}
\newcommand{\fkH}{\mathfrakcommand{H}}
\newcommand{\fkI}{\mathfrakcommand{I}}
\newcommand{\fkJ}{\mathfrakcommand{J}}
\newcommand{\fkK}{\mathfrakcommand{K}}
\newcommand{\fkL}{\mathfrakcommand{L}}
\newcommand{\fkM}{\mathfrakcommand{M}}
\newcommand{\fkN}{\mathfrakcommand{N}}
\newcommand{\fkO}{\mathfrakcommand{O}}
\newcommand{\fkP}{\mathfrakcommand{P}}
\newcommand{\fkQ}{\mathfrakcommand{Q}}
\newcommand{\fkR}{\mathfrakcommand{R}}
\newcommand{\fkS}{\mathfrakcommand{S}}
\newcommand{\fkT}{\mathfrakcommand{T}}
\newcommand{\fkU}{\mathfrakcommand{U}}
\newcommand{\fkV}{\mathfrakcommand{V}}
\newcommand{\fkW}{\mathfrakcommand{W}}
\newcommand{\fkX}{\mathfrakcommand{X}}
\newcommand{\fkY}{\mathfrakcommand{Y}}
\newcommand{\fkZ}{\mathfrakcommand{Z}}
\newcommand{\fka}{\mathfrakcommand{a}}
\newcommand{\fkb}{\mathfrakcommand{b}}
\newcommand{\fkc}{\mathfrakcommand{c}}
\newcommand{\fkd}{\mathfrakcommand{d}}
\newcommand{\fke}{\mathfrakcommand{e}}
\newcommand{\fkf}{\mathfrakcommand{f}}
\newcommand{\fkg}{\mathfrakcommand{g}}
\newcommand{\fkh}{\mathfrakcommand{h}}
\newcommand{\fki}{\mathfrakcommand{i}}
\newcommand{\fkj}{\mathfrakcommand{j}}
\newcommand{\fkk}{\mathfrakcommand{k}}
\newcommand{\fkl}{\mathfrakcommand{l}}
\newcommand{\fkm}{\mathfrakcommand{m}}
\newcommand{\fkn}{\mathfrakcommand{n}}
\newcommand{\fko}{\mathfrakcommand{o}}
\newcommand{\fkp}{\mathfrakcommand{p}}
\newcommand{\fkq}{\mathfrakcommand{q}}
\newcommand{\fkr}{\mathfrakcommand{r}}
\newcommand{\fks}{\mathfrakcommand{s}}
\newcommand{\fkt}{\mathfrakcommand{t}}
\newcommand{\fku}{\mathfrakcommand{u}}
\newcommand{\fkv}{\mathfrakcommand{v}}
\newcommand{\fkw}{\mathfrakcommand{w}}
\newcommand{\fkx}{\mathfrakcommand{x}}
\newcommand{\fky}{\mathfrakcommand{y}}
\newcommand{\fkz}{\mathfrakcommand{z}}


\title{Abstract Domains and Solvers for Sets Reasoning}

\author{Arlen Cox, Bor-Yuh Evan Chang\inst{1}, Huisong Li\inst{2}, Xavier Rival\inst{2}}

\institute{University of Colorado Boulder$^1$
  \qquad
  Inria/CNRS/ENS Paris/PSL*$^2$}

\begin{document}
    
\maketitle
\begin{abstract}
  Software static analysis tools often need to perform set reasoning in
  order to track sets of program values, pointers and memory addresses.
  Relations among sets and elements need to be tracked during the analysis
  to enable the verification of some classes of program semantic properties.
  We formalize an {\em abstract domain view} of set reasoning, in the sense
  of abstract interpretation, which integrates naturally into interpretation
  based static analyzers.
  We construct basic set abstract domains relying on efficient specialized
  data-structures and propose techniques to lift set abstract domains into
  more expressive or more efficient ones.
  We also show that readily available solvers can be turned into set
  abstract domains.
  We report on the design and implementation of QUICr, a library of set
  abstract domains and assess the efficiency and precision of domains
  built from specialized structures and solvers.
\end{abstract}

%-----------------------------------------%
% Colors                                  %
%-----------------------------------------%
\newrgbcolor{mblue}{        0.1   0.1   0.5 }
\newrgbcolor{mbrown}{       0.57  0.4   0.3 }
\newrgbcolor{mcyan}{        0.1   0.1   0.5 }
\newrgbcolor{mgreen}{       0.2   0.5   0.2 }
\newrgbcolor{mmidgray}{     0.65  0.65  0.65}
\newrgbcolor{morange}{      0.8   0.45  0.45}
\newrgbcolor{mpurple}{      0.5   0.1   0.5 }
\newrgbcolor{mred}{         1     0     0   }
\newrgbcolor{mhigray}{      0.21  0.21  0.21}
\newrgbcolor{mgray}{        0.5   0.5   0.5 }
\newrgbcolor{mmidgreen}{    0.3   0.8   0.4 }
\newrgbcolor{mlightgray}{   0.85  0.85  0.85}
\newrgbcolor{mlightblue}{   0.7   0.85  1.00}
\newrgbcolor{mlightgreen}{  0.7   1.00  0.85}
\newrgbcolor{mlightpurple}{ 1.00  0.7   1.00}
\newrgbcolor{mlightred} {   1.00  0.7   0.7 }
\newrgbcolor{mlightyellow}{ 1.00  1.00  0.6 }
\newrgbcolor{mulightgray}{  0.98  0.98  0.98}
\newrgbcolor{mulightblue}{  0.9   0.95  1.00}
\newrgbcolor{mulightgreen}{ 0.9   1.00  0.95}
\newrgbcolor{mulightpurple}{1.00  0.9   1.00}
\newrgbcolor{mulightred} {  1.00  0.85  0.85}
\newrgbcolor{mulightyellow}{1.00  1.00  0.87}
\newrgbcolor{mvlightblue}{  0.8   0.9   1.00}
\newrgbcolor{mvlightgreen}{ 0.8   1.00  0.9 }
\newrgbcolor{mvlightpurple}{1.00  0.8   1.00}
\newrgbcolor{mvlightred} {  1.00  0.85  0.85}
\newrgbcolor{mvlightyellow}{1.00  1.00  0.73}
\newrgbcolor{myellow}{      1.00  1.00  0.30}
\newrgbcolor{mwhite}{       0.99  0.99  0.99}

%-----------------------------------------%
% Typing and abbreviations                %
%-----------------------------------------%
% Standard Abbrevs
\newcommand{\eg}{e.g.\xspace}
\newcommand{\ie}{i.e.\xspace}
\newcommand{\resp}{resp.\xspace}
% Removing
\newcommand{\commentout}[1]{}
% Project names
\newcommand{\memcad}{{MemCAD}\xspace}

%-----------------------------------------%
% Math standard stuffs                    %
%-----------------------------------------%
% Abstraction
\newcommand{\abs}[1]{{#1}^{\sharp}}
% Booleans
\newcommand{\true}{\mathbf{true}}
\newcommand{\false}{\mathbf{false}}
% Definitions
\newcommand{\isdef}{::=}
% Logical relations
\newcommand{\logor}{\mathrel{\vee}}
\newcommand{\logand}{\mathrel{\wedge}}
\newcommand{\suplus}{\mathrel{\uplus}}
% Powerset
\newcommand{\partsof}[1]{\mcP(#1)}
% Semantics
\newcommand{\sem}[1]{\llbracket #1 \rrbracket}
\newcommand{\asem}[1]{\abs{\llbracket #1 \rrbracket}}

%-----------------------------------------%
% Inserting Tikz pictures                 %
%-----------------------------------------%
% Including pictures
\newcommand{\tikzpics}[2]{\scalebox{#1}{\input{tkz-#2.tex}}} % with scale
\newcommand{\tikzpic}[1]{\tikzpics{1}{#1}}                   % without scale
% Putting a formula
\newcommand{\tkzputform}[2]{\node[] at (#1){\ensuremath{#2}};}
\newcommand{\tkzputrghform}[2]{
  \node[label distance=-10pt,inner sep=0pt,outer sep=0pt,
    minimum height=0pt,minimum width=0pt,
    outer xsep=0pt,outer ysep=0pt,inner xsep=0pt,inner ysep=0pt,
    label=right:\ensuremath{#2}] at (#1){};}
\newcommand{\tkzputlftform}[2]{
  \node[label distance=-10pt,inner sep=0pt,outer sep=0pt,
    minimum height=0pt,minimum width=0pt,
    outer xsep=0pt,outer ysep=0pt,inner xsep=0pt,inner ysep=0pt,
    label=left:\ensuremath{#2}] at (#1){};}
% Putting a piece of text
\newcommand{\tkzputtext}[2]{\node[] at (#1){#2};}
\newcommand{\tkzputrghtext}[2]{
  \node[label distance=-10pt,inner sep=0pt,outer sep=0pt,
    minimum height=0pt,minimum width=0pt,
    outer xsep=0pt,outer ysep=0pt,inner xsep=0pt,inner ysep=0pt,
    label=right:#2] at (#1){};}
\newcommand{\tkzputlfttext}[2]{
  \node[label distance=-10pt,inner sep=0pt,outer sep=0pt,
    minimum height=0pt,minimum width=0pt,
    outer xsep=0pt,outer ysep=0pt,inner xsep=0pt,inner ysep=0pt,
    label=left:#2] at (#1){};}
% Pointers
\newcommand{\tkptrdot}[1]{\draw (#1) [fill=black] circle (0.05);}

%-----------------------------------------%
% Programs                                %
%-----------------------------------------%
% Variables
\newcommand{\ttvar}[1]{\texttt{#1}}
\newcommand{\cvar}[1]{\ensuremath{\texttt{#1}}}
\newcommand{\varc}{\ttvar{c}}
\newcommand{\vard}{\ttvar{d}}
\newcommand{\varg}{\ttvar{g}}
\newcommand{\varh}{\ttvar{h}}
\newcommand{\vari}{\ttvar{i}}
\newcommand{\varl}{\ttvar{l}}
\newcommand{\varm}{\ttvar{m}}
\newcommand{\varn}{\ttvar{n}}
\newcommand{\varp}{\ttvar{p}}
\newcommand{\varr}{\ttvar{r}}
\newcommand{\vars}{\ttvar{s}}
\newcommand{\vart}{\ttvar{t}}
\newcommand{\varu}{\ttvar{u}}
\newcommand{\varv}{\ttvar{v}}
\newcommand{\varx}{\ttvar{x}}
\newcommand{\vary}{\ttvar{y}}
% Pointers
\newcommand{\nullptr}{\textbf{0x0}}

%-----------------------------------------%
% Examples                                %
%-----------------------------------------%
\newcommand{\naddr}[1]{\texttt{n}_{#1}} % Node addresses

%-----------------------------------------%
% Set abstraction                         %
%-----------------------------------------%
% set of all set variables
\newcommand{\setvars}{\bbX_{\bf s}}
% set variables (in case we change font)
\newcommand{\setvw}{W}
\newcommand{\setvx}{X}
\newcommand{\setvy}{Y}
\newcommand{\setvz}{Z}
% set of all elements
\newcommand{\values}{\bbV}
% set of concrete states
\newcommand{\states}{\bbS}
% symbolic sets
\newcommand{\symsets}{\bbC}
% abstract domain (generic)
\newcommand{\adom}{\abs{\bbD}}     % abstract domain
\newcommand{\astate}{\abs{\state}} % generic abstrct element
\newcommand{\gammadom}{\gamma}     % concretization function
% extremal elements
\newcommand{\adombot}{\bot_{\adom}}
\newcommand{\adomtop}{\top_{\adom}}
% abstract operations
\newcommand{\adomisbot}{\textbf{isbot}_{\adom}}
\newcommand{\adomforget}{\textbf{forget}_{\adom}}
\newcommand{\adomassume}{\textbf{assume}_{\adom}}
\newcommand{\adomprove}{\textbf{prove}_{\adom}}
\newcommand{\adomjoin}{\textbf{join}_{\adom}}
\newcommand{\adomwiden}{\textbf{widen}_{\adom}}
\newcommand{\adomisle}{\textbf{is\_le}_{\adom}}
\newcommand{\adommeet}{\textbf{meet}_{\adom}}

%-----------------------------------------%
% single abstract state
\renewcommand{\abstract}{\Sigma}
% set of all abstract states
\newcommand{\abstracts}{\textsc{AbsStates}}
% cardinality of a set
\newcommand{\card}[1]{|{#1}|}
% complement of a set
\newcommand{\comp}[1]{{#1}^{\mathrm{c}}}
% individual set variables: W, X, Y, or Z
% set of all set variables
%\newcommand{\setvars}{\textsc{SetVars}}
% concrete state
\newcommand{\state}{\sigma}
% set of all concrete states
%\newcommand{\states}{\textsc{States}}
% set of all values
%\newcommand{\values}{\textsc{Vals}}
% concrete set c
% syntax of a logical constraint L
% syntax of a set expression E
% power set of a set
\newcommand{\powerset}[1]{\mathcal{P}\left({#1}\right)}


\newcommand{\defeq}{\stackrel{\textrm{\tiny def}}{=}}

%1. Introduction (1.5 pages)
%- Goals:
%- Set abstractions serve a different purpose than SAT/SMT/MC
%- Inference vs Entailment
%- Explicit vs Implicit control flow
%- Handling of quantifiers
%- Forward vs backward analysis
%- Modern set abstractions offer performance and precision
%- Set abstractions are useful for a wide variety of problems
\section{Introduction}
\label{s:1:intro}
The verification of program properties that
involve data structures is a challenging problem~\cite{jahob:thesis:07,compass:popl:11,fixbag:cav:11,celia:vmcai:12,ab:ecoop:13,hoo:14:sas,memcad:15:sas}.
One key reason for this is that
if a data structure is unbounded, there is a potentially unbounded number of constraints on its elements.  Since these constraints often affect important properties such as memory safety~\cite{memcad:15:sas}, functional correctness~\cite{fixbag:cav:11}, or basic program behavior~\cite{hoo:14:sas}, it is vital to develop techniques for efficiently reasoning about relationships between unbounded numbers of elements.

This paper focuses on the use of set constraints to reason about unbounded collections of elements.  Set constraints can be used to dynamically partition data structures, correlate collections of elements with one another, or determine analysis case splits.  They are useful for representing data and pointers relationships in structures such as maps, graphs, lists, sets, and arrays.  They can be combined with other techniques such as separation logic~\cite{hoo:14:sas,memcad:15:sas} and numerical analyses~\cite{quicr:cav:14} to enhance those analyses.

%% xr: need to adjust overview, there is some repetition now
For example, consider the program in Figure~\ref{fig:intro-example}
that copies one map on top of another.
Within the loop, there is a complex relationship between the sets of
keys of \ttvar{src} and \ttvar{dst}.
At the specified point, the keys of \ttvar{src} can be partitioned into
three parts.
The keys already visited $X_v$ by the loop, the element currently being
visited by the loop $\{\varx\}$, and the keys not visited $X_n$ by
the loop.
The keys of \ttvar{dst} can be partitioned into those originally in
\ttvar{dst} that have not been overwritten, and those $X_v$ that have
been overwritten or added from \ttvar{src}.
This set reasoning allows precise symbolic tracking of the provenance
of map partitions.

\begin{figure}[tb]
  \newbox\exprogbox
  \begin{lrbox}{\exprogbox}
    \begin{minipage}[t][1cm][b]{0.4\textwidth}
      \begin{lstlisting}[language=python]
def extend(dst, src):
  for x in src:
    # invariant point
    dst[x] = src[x]
      \end{lstlisting}
    \end{minipage}
  \end{lrbox}
  \newbox\exproginv
  \begin{lrbox}{\exproginv}
    \begin{minipage}[t][0.9cm][b]{0.4\textwidth}
      \begin{align*}
        \exists X_v, X_n. \; keys(\ttvar{src})
        & = X_v \suplus \{\varx\} \suplus X_n
        \\
        {} \logand keys(\ttvar{dst})
        & = (keys(\ttvar{dst})_0 \setminus X_v) \suplus X_v
      \end{align*}
    \end{minipage}
  \end{lrbox}
  \centering
  \subfigure[Function to copy all keys and values from map \ttvar{src}
  onto map \ttvar{dst}]{\usebox\exprogbox}
  \quad
  \subfigure[Set abstraction at \texttt{\textit{invariant point}}]{%
    \usebox\exproginv}
  \caption{Set constraints can relate portions of data structures}
  \label{fig:intro-example} \label{f:1:intro}
\end{figure}

This paper focuses on abstractions for states described by the logic for \emph{symbolic sets}.  The logic consists of a Boolean algebra over the set variables with singleton sets. We find that this subset is sufficiently large to be useful and we
believe that it serves as a good starting point for extensions to the
logic, such as reasoning about explicit set contents or more precise
cardinality.

However, despite the fact that we are not reasoning about the values
contained in sets or complex cardinalities, Boolean algebras, by
themselves, are challenging for invariant generation.
Naive approaches such as saturation and pattern matching rarely work
without complex heuristics~\cite{fixbag:cav:11,ab:ecoop:13}.
It is unavoidable that the worst-case time for precise invariant
generation will be exponential because of the Boolean algebra.
However, it is desirable that invariant generation should be efficient
in the common cases, and unlike systems that involve complex heuristics,
lose precision only in understandable and predictable ways.

% AC: seems like this could be tightened up and combined with the next paragraph
In this paper we aim to design scalable, precise, and predictable abstractions for symbolic sets by combining new abstract domains with performance/precision-enhancing functors that lift existing set abstractions to new set abstractions.  Specifically, we make the following contributions:
%This includes abstractions based on binary decision diagrams,
%satisfiability modulo theories, and linear set constraints.
%Additionally, it includes performance and precision enhancing combinators for tracking
%singleton sets, handling equality, and doing dynamic variable packing.
%These abstractions and combinators exist within a general framework so
%that additional abstractions and combinators can be easily added.
%This framework is available within the QUICr library, which is now
%used by two research analyzers.
%
%In this paper we make the following contributions.
\begin{compactitem}
\item We define a common interface for symbolic set abstractions that is designed to meet the needs of static analyzers (Section~\ref{sec:logic-and-set-abstraction}).
%
\item Using specialized data structures, we construct a battery of symbolic set abstract domains and performance/precision-enhancing functors designed to target real-world data structure verification problems (Section~\ref{sec:constructed}).
%
\item We adapt an off-the-shelf satisfiability-modulo-theories solver to the set abstraction interface (Section~\ref{sec:solvers}).
%
\item We compare abstractions for symbolic sets, finding that, while specialized abstractions are preferable, binary decision diagrams lifted with dynamic packing is a good compromise in scalability, performance, and predictability (Section~\ref{sec:evaluation}).
%
\end{compactitem}

%2. Overview (2 pages)
%-> a couple of examples, showing very intuitively (possibly just with pictures)
%the need for set abstractions
%Possibly:
%- a quick picture of HOO
%- a quick picture of Huisong's work
\section{Overview}
\label{s:2:over}
In this section, we present two static analyses that make use of set
reasoning in order to compute high level semantic properties of
programs.
These analyses rely on abstract interpretation~\cite{cc:popl:77} and
on an abstraction of program states, that describes data structures
and their contents.
An abstract domain defines a set of predicates that an analysis may
use, as well as operators to over-approximate the effect of program
behaviors on these predicates, and their implementation.

\paragraph{Inferrence of properties of open objects.}
Dynamic programming languages such as JavaScript feature {\em open objects}
that support dynamic addition and deletion of attributes and iteration over
them.
The analysis presented in~\cite{hoo:14:sas} computes relations
among objects, so as to verify programs as the piece of code of
Figure~\ref{f:1:intro}.
To achieve this, it infers relations between the sets of attributes
of distinct objects.
Objects have an unbounded number of attributes, thus the analysis requires
some abstraction over the attributes and their contents.
\newcommand{\varsrc}{\cvar{src}}
\newcommand{\vardst}{\cvar{dst}}
\begin{figure}[t]
  \newcommand{\picscale}{0.82}
  \begin{center}
    \tikzpics{\picscale}{hoo-inv}
  \end{center}
  \caption{Open objects and their abstraction}
  \label{f:2:hoo}
\end{figure}
Figure~\ref{f:2:hoo} represents a very simplified state, at the loop
head and after two iteration (thus two fields were copied).
We focus on the set of attributes of each object, and ignore their
contents (which could be described using similar techniques).
To abstract precisely the relations between the attributes of both objects
(\ie, in this case, copied attributes are common to both objects), we need
to describe the fields of each object as the union of a series of sets of
attributes, and to express relations among these sets.
The first purpose of the set abstract domain is to represent such set
relations.
The right hand side of Figure~\ref{f:2:hoo} depicts such an abstract
state, where \( X_n, X_r, X_v \) stand for sets of attributes, which are
made explicit, in the case of the left hand side concrete state.

Moreover, to infer these invariants, the analysis needs to reason
both about object structures and about attributes sets:
\begin{asparaitem}
\item Initially, no set relation should be assumed, and the fields of
  each object should be associated to a set of attributes;
\item When the analysis enters the body of the loop, it needs to
  {\em single out} attribute \( \mathtt{x} \), thus, to replace set
  \( X_v \) by \( X_v \suplus \{ \mathtt{x} \} \) (which produces the
  equalities of Figure~\ref{f:2:hoo});
\item When it exits the loop, the analysis should {\em generalize}
  both the object and set constraints abstractions, which requires
  to {\em eliminate} singleton \( \{ \mathtt{x} \} \) from the
  equations (it is visible only in the loop body) and to synthesize
  a new, more general collection of constraints.
\end{asparaitem}
To allow these steps, the set abstraction should provide basic operations
over set predicates, including (1) the addition of a set constraint, (2)
the proving of a set constraint, (3) the removal of a set variable and
(4) the generalization of two set abstract states.

\paragraph{Shape analysis in presence of unstructured sharing.}
The shape analysis for data-structures with unbounded sharing presented
in~\cite{memcad:15:sas} relies on separation logic~\cite{r:lics:02} to
describe memory states, and on inductive definitions to summarize
unbounded structures such as lists.
Unstructured sharing is very challenging as it cannot be described using
conventional inductive definitions.
Figure~\ref{f:3:memcad} displays the representation of a three nodes
graph using an adjacency list data-structure in the left.
To summarize such a structure using inductive predicates in separation
logic, \cite{memcad:15:sas} proposes to augment the list inductive
predicates with set information, which express where edges may point to.
This representation is shown in the right of Figure~\ref{f:3:memcad} in
a form where the first node is kept materialized.
It both asserts that edges of that node as well as edges from other nodes
point to the address of a valid node, namely an element of \( \{ \naddr{0}
\} \suplus \mcE \).
\begin{figure}[t]
  \newcommand{\picscale}{0.9}
  \tikzpics{\picscale}{memcad-inv}
  \caption{Summarization of an adjacency list-based graph representation}
  \label{f:3:memcad}
\end{figure}
The analysis of~\cite{memcad:15:sas} introduces a summary predicate
\( \textbf{graph}( \naddr{0}, \mcN ) \) where \( \naddr{0} \) is the
address of the first node and \( \mcN \) the set of all node addresses.
This predicate is defined by induction over the ``backbone'' of the
structure, and fully takes into account the property that all
\( \fldnext \) edges point to a valid node address in \( \naddr{0} \).
Henceforth, abstract states comprise both a {\em memory} part (which
consists of a formlula in separation logic with inductive predicates)
and a {\em set abstraction}.

To compute such summaries, the analysis needs to perform similar
operations as the analysis for open objects, in order to add set
constraints to the set abstract state, prove set constraints,
remove set variables and generalize abstract states.

% \paragraph{Need for set reasoning.}
% Both analyses consist of a composite abstract domain, that comprises a
% {\em structural abstract domain} describing the shape of structures, and
% of a {\em set abstract domain} expressing relations among set and value
% (pointers, field names, contents...) entities.


%3. Set Abstraction Problem (1 page)
%-> set up problem framework / why abstract domain vs other things?
%- Boolean Algebra
%- Cardinality/Value considerations
%- Forward analysis to be use as a subcomponent of other analyses
%- Abstract domain (implicit vs explicit control flow)
%- Inference (vs BAPA)
\section{Logic and Set Abstraction}
\label{sec:logic-and-set-abstraction} \label{s:3:abs}
We now define the elements and operators of a set abstract domain that meets the needs of all the analyses shown in Section~\ref{s:2:over}.

\paragraph{Concrete states.}
In this paper, we use symbols $\setvw$, $\setvx$, $\setvy$, and $\setvz$ as set variables
and let $\setvars$ represent the set of all such variables.
We are interested in purely symbolic set relations, and do not make any
assumption on the type of the set elements (in practice these are pointers
or scalars).
We let \( \values \) denote the set of all these elements.
A concrete state is a function \( \state: \setvars \rightarrow
\partsof{\values} \).
We write \( \states \) for the set of such elements.

\paragraph{Symbolic sets.}
Before we set up the signature of abstract domains, we fix a language
of set predicates, that will be used as a basis for abstract elements,
and for the communication with the set abstract domain.
\begin{definition}[Symbolic Sets]
  \label{d:1:symsets}
  {\em Symbolic sets} are defined by the grammar:
  \begin{align*}
    L (\in \symsets) ::=
    & \ L \wedge L \
    | \ E \subseteq E \
    | \ \card{X} = 1 \
    | \ \top \
    | \ \bot
    & E ::=
    & \ \emptyset \ | \ X \ | \ \comp{E} \ | \ E \cup E \ | \ E \suplus E
  \end{align*}
\end{definition}
The meaning of these constraints is straightforward, but we give a formal
definition in Figure~\ref{f:4:symsets} for clarity.
A model of a set expression $E$ is a concrete state $\state$ and a set
of concrete values $c$.
A model of a logical expression $L$ is a concrete state $\state$.
The concretization is $\gamma(L) = \{\ \sigma \ |\  \sigma \models L\ \}$
and we use $\aform{L}$ for abstract states with the same concretization.
\begin{figure}[t]
  \begin{align*}
    & \state, c \models \emptyset \textrm{ iff } c = \emptyset
    \qquad
    \state, c \models X \textrm{ iff } c = \state(X)
    \qquad \state, c \models \comp{E}
    \textrm{ iff }
    \state, c' \models{E}
    \textrm{ and } \forall v \in \values.
    \; v \in c \Leftrightarrow v \not\in c'
    \\
    & \state, c \models E_1 \cup E_2
    \textrm{ iff }
    \state, c_1 \models E_1
    \textrm{ and } \state, c_2 \models E_2
    \textrm{ and }
    \forall v \in \values. \; v \in c \Leftrightarrow v \in c_1 \vee
    v \in c_2
    \\
    & \state, c \models E_1 \suplus E_2
    \textrm{ iff }
    \state, c_1 \models E_1
    \textrm{ and } \state, c_2 \models E_2
    \textrm{ and }
    \forall v \in \values. \; v \in c \Leftrightarrow v \in c_1 \vee
    v \in c_2
    \textrm{ and } c_1 \cap c_2 = \emptyset
    \\
    & \state \models L_1 \wedge L_2
    \textrm{ iff }
    \state \models L_1 \textrm{ and } \state \models L_2
    \qquad
    \state \models \card{E} = 1
    \textrm{ iff }
    \state, c \models E \textrm{ and } \exists v \in \values. \; c = \{ v \}
    \\
    & \state \models E_1 \subseteq E_2
    \textrm{ iff }
    \state, c_1 \models E_1 \textrm{ and } \state, c_2 \models E_2
    \textrm{ and } \forall v \in \values. \; v \in c_1 \rightarrow v \in c_2
    \qquad \state \models \top
    \qquad \state \not\models \bot
  \end{align*}
  \caption{Symbolic set constraint language}
  \label{f:4:symsets}
\end{figure}
We shall also use the following derived logical forms for simplicity:
\[
E_1 \cap E_2 \defeq \comp{(\comp{E_1} \cup \comp{E_2})}
\qquad
E_1 = E_2 \defeq E_1 \subseteq E_2 \wedge E_2 \subseteq E_1
\qquad
E_1 \setminus E_2 \defeq E_1 \cap \comp{E_2}
\]

\paragraph{Set abstraction.}
A {\em set abstract domain} is defined by a set of {\em abstract elements}
\( \adom \) which describe the family of logical properties it can
express and a concretization function \( \gammadom: \adom \rightarrow
\partsof{\states} \) that maps each element of \( \adom \) into the set
of concrete states that satisfy it.
Abstract elements are characterized by
(1) the symbolic sets they describe and
(2) their machine representation.
The latter is usually very different from the formulas, and will be
discussed in Section~\ref{s:4:domains}.
\begin{example}[(Non-)Emptiness set domain]
  \label{ex:1:mt}
  A very basic example of such a domain is the {\em (non-)emptiness} domain
  that comprises the following elements:
  \begin{compactitem}
  \item \( \bot \), which denotes the unsatisfiable abstract constraint
    (\ie, \( \gammadom( \bot ) = \emptyset ) \));
  \item the functions from \( \setvars \) into \( \{ [=\emptyset],
    [\not=\emptyset], \top \} \), which map each set variable into
    its emptiness value.
  \end{compactitem}
  For instance, \( \{ \setvx \mapsto \top; \setvy \mapsto [=\emptyset] \} \)
  stands for \( \aform{\setvy \subseteq \emptyset} \) and concretizes into
  \( \gamma( \setvy \subseteq \emptyset ) \).
\end{example}

\paragraph{Operations over Set Abstractions.}
We now formalize the main operations and logical elements needed so that
we can use a set abstract \( \adom \) domain for either of the static
analyses shown in Section~\ref{s:2:over}.
\begin{asparaitem}
\item \emph{Basic logical elements.}
  Static analyses typically start with an unconstrained state.
  This is indicated by a \( \adomtop \in \adom \) element with full
  concretization, \ie, \( \gammadom( \adomtop ) = \states \).
  Similarly, the abstract element \( \adombot \in \adom \) should describe
  the unsatisfiable abstract constraint (\ie, \( \gammadom( \adombot ) =
  \emptyset \)).
  In Example~\ref{ex:1:mt}, \( \adombot \) is \( \bot \) and \( \adomtop \)
  is \( \lambda (x \in \setvars) \cdot \top \).
  Moreover, a static analysis often has to determine if an abstract state
  describes unsatisfiable constraints.
  Thus, \( \adom \) should provide an operator \( \adomisbot: \adom
  \rightarrow \{ \true, \false \} \) such that \( \adomisbot( \astate )
  = \true \Longrightarrow \gammadom( \astate ) = \emptyset \).

\item \emph{Forgetting a set variable.}
  Static analysis tools drop set variables that become redundant.
  In the open object example of Section~\ref{s:2:over}, this occurs when
  the singleton symbol is eliminated at the end of the loop.
  To do this, we require the set abstract domain \( \adom \) to provide
  an operator \( \adomforget: \adom \times \setvars \rightarrow \adom \)
  that discards a symbol from the abstract state.
  % forget soundness property

\item \emph{Assuming set constraints.}
  As noted in Section~\ref{s:2:over}, an important set reasoning step
  {\em restricts an abstract state with set constraints}, thus set
  domain \( \adom \) should provide an operator \( \adomassume: \adom
  \times \symsets \rightarrow \adom \), which conservatively represents
  a constraint into an abstract state, \ie ensures that, for all
  \( \astate, L \), \( \gammadom( \astate ) \cap \gamma( L ) \subseteq
  \gammadom( \adomassume( \astate, L ) ) \).
  Note that this operator also makes use of the symbolic set language
  of Definition~\ref{d:1:symsets} in order to describe constraints
  communicated to the domain.
  
\item \emph{Verifying set constraints.}
  Similarly, set reasoning should allow {\em verifying
    set constraints}, thus the set domain \( \adom \) should provide an operator
  \( \adomprove: \adom \times \symsets \rightarrow \{ \true, \false \} \),
  which conservatively attempts to verify that a symbolic set constraint
  holds under some abstract states, \ie ensures that, for all \( \astate,
  L \), \( \adomprove( \astate, L ) = \true \) implies that
  \( \gammadom( \astate ) \subseteq \gamma( L ) \).

\item \emph{Generalizing set abstractions.}
  The analysis of loops is commonly based on the computation of abstract
  post-fixpoints~\cite{cc:popl:77}, thus \( \adom \) should provide
  sound over-approximation of the union of sets concrete states.
  In the logical point of view, this amounts to computing a common weakening
  for two abstract constraints.
  This is performed by an operator \( \adomjoin: \adom \times \adom
  \rightarrow \adom \) such that, for all \( \astate_0, \astate_1 \),
  \( \gammadom( \astate_0 ) \cup \gammadom( \astate_1 ) \subseteq
  \gammadom( \adomjoin( \astate_0, \astate_1 ) ) \).
  Widening operator \( \adomwiden \) should satisfy the same property
  and ensure termination of any sequence of abstract iterates.

\item \emph{Deciding entailment over set abstractions.}
  Finally, the operator \( \adomisle: \adom \times \adom \rightarrow \{ \true,
  \false \} \) conservatively decides implication among abstract set
  constraints (by ensuring that \( \adomisle( \astate_0, \astate_1 ) =
  \true \Longrightarrow \gammadom( \astate_0 ) \subseteq \gammadom(
  \astate_1 ) \)), and allows verifying the convergence of abstract
  iterates.
\end{asparaitem}
%ac: Should we include the rename operator here?  It seems critical to a lot of what we do, and while we consider it fairly trivial for domains, it is a non-trivial, expensive operation for SMT and solver domains.


%4 Constructed set abstractions (5 pages)
%-> search for efficient structures and algorithms associated to them
%4.1 Lin
%4.2 QUICr
%4.3 BDD
%4.4 EQ functor
%4.5 Packing functor
%- Why packing doesn't work for BDDs in model checking, but does work here.
\section{Constructed Set Abstractions}
\label{sec:constructed} \label{s:4:domains}
An abstract domain is defined by a class of set constraints and their machine
representation, and abstract operations following the signature given in
Section~\ref{s:3:abs}.
In this section, we introduce three basic set abstract domains (respectively
based on linear constraints, QUIC graphs, and BDDs) and two set abstract
domain functors, that lift a set domain into another, more expressive or
efficient one.

\subsection{Linear Set Constraints}
\label{s:4:1:lin}
\newcommand{\adomlin}{\adom_{\zcl}}
\newcommand{\gammalin}{\gamma_{\zcl}}
\paragraph{Abstract elements and their concretization.}
Our first set abstract domain relies on {\em linear} set equality
constraints, of the form \( \aform{\setvx = \{ y_0, \ldots, y_k \} \suplus
  \setvz_0 \suplus \ldots \suplus \setvz_l} \).
The advantage of such constraints is to provide a rather straightforward
normalization of the representation of constraints.
Note they also include emptiness constraints.
Our implementation of abstract domain \( \adomlin \) describes three kinds
of constraints:
\begin{compactitem}
\item acyclic {\em linear} constraints of the form \( \aform{\setvx =
    \setvy_0 \suplus \ldots \suplus \setvy_k \suplus \setvz_0 \suplus \ldots
    \suplus \setvz_l} \), where \( \setvy_0, \ldots, \setvy_k \) are
  singletons (in our implementation, each variable may appear at most
  {\em once} as the left hand side of such a constraint, to enable
  normalization);
\item inclusion constraints of the form \( \aform{\setvy \subseteq \setvx} \);
\item equality constraints of the form \( \aform{\setvy = \setvx} \).
\end{compactitem}
Thus, an element of \( \adomlin \) is either \( \bot \) or a conjunction of
such constraints.
The associated concretization \( \gammalin: \adomlin \rightarrow
\partsof{\states} \) is of the same form as that of the symbolic sets
language of Definition~\ref{d:1:symsets} (thus, we do not formalize it
in full details).
The machine representation utilizes persistent dictionaries, that stand
for functions over a finite domain.
This reduces basic queries for facts (such as: ``does abstract state
\( \astate \) entail that \( \setvx \subseteq \setvy \suplus \setvz
\)~?'') to dictionary searches.

\paragraph{Abstract operators.}
The core algorithm of \( \adomlin \) normalizes an abstract values by
expanding nested linear constraints:
for instance, \( \aform{\setvx_0 = \setvx_1 \suplus \setvx_2 \wedge
  \setvx_1 = \setvx_3 \suplus \setvx_4} \) is rewritten into
\( \aform{\setvx_0 = \setvx_2 \suplus \setvx_3 \suplus \setvx_4 \wedge
  \setvx_1 = \setvx_3 \suplus \setvx_4} \) at the machine representation
level.
This process terminates as constraints represented in \( \adomlin \)
do not contain cycles.
It is performed incrementally by all abstract operations.

Abstract operations \( \adomisbot, \adomassume, \adomprove \) are all
made very fast by this normalization.
Operation \( \adomforget \) simply drops all constraints that involve
a given set variable.
Finally, \( \adomjoin \) and \( \adomwiden \) need to {\em generalize}
constraints:
\begin{example}
  Let us assume that \( \astate_0 \) (\resp, \( \astate_1 \)) stands for
  the set of constraints \( \aform{\setvx_0 = \setvx_1 \suplus \setvx_2
    \wedge \setvx_3 = \emptyset} \) (\resp, \( \aform{\setvx_0 = \setvx_1
    \suplus \setvx_2 \suplus \setvx_3} \)).
  Then \( \adomjoin( \astate_0, \astate_1 ) \) returns an element that
  represents the constraint \( \aform{\setvx_0 = \setvx_1 \suplus \setvx_2
    \suplus \setvx_3} \).
\end{example}
\memcad~\cite{memcad:15:sas} relies on \( \adomlin \) to represent set
constraints since it mainly needs to express constraints over set
partitions.
In the other hand, \( \adomlin \) is not adapted to the precise
description of non disjoint unions.

\subsection{QUIC graphs}
\label{s:4:2:quic}

A QUIC graph~\cite{ab:ecoop:13} is a directed hypergraph data structure used to represent relational set constraints.  Each edge in the hypergraph corresponds to a subset constrtaint and each hypergraph is a conjunction of subset constraints where each constraint is of the form: $\aform{X_1 \cap \ldots \cap X_n \subseteq Y_1 \cup \ldots \cup Y_m}$.  Each variable can also be constrained to be a singleton with constraints such as $\aform{|X| = 1}$.  The concretization $\gamma_q : \adom_q \rightarrow \partsof{\states}$ is of the same form as that of the symbolic sets language of Definition~\ref{d:1:symsets}.

% - algorithms main ideas
QUIC graphs is designed for efficiently performing two operations:
(1) $\adomforget$, which matches edges containing the symbol to be
forgotten with each other to produce new edges without that symbol; and
(2) content reasoning, which is not a design goal for symbolic sets.
The $\adomjoin$ and $\adomwiden$ operations are primarily based on
saturation heuristics.
They keep common conjunctions from both arguments.
To aid this process, they use a form of saturation that produces new
conjuncts based on pattern matches.
A sufficiently large set of patterns must be provided to attain precision,
but additional patterns increase the cost of joins.
% - example
\begin{example}[QUIC graph join] \label{ex:Qjoin}  Consider the following join operation:
    \begin{align*}
      \astate_0 &= \aform{X_1 \subseteq X_3 \logand X_3 \subseteq X_2} &
      \astate_1 &= \aform{X_1 \subseteq X_4 \logand X_4 \subseteq X_2} &
      \adomjoin(\astate_0, \astate_1 )
    \end{align*}
    There is an obvious result: $\aform{X_1 \subseteq X_2}$.  Whether or not QUIC graphs derives this result or top is determined by the pattern matches that are installed.  If the pattern takes $\aform{X_a \subseteq X_b \logand X_b \subseteq X_c}$ and generates $\aform{X_a \subseteq X_c}$ is installed, it will first apply the pattern to both sides and then keep common conjuncts, getting the desired result.  Without that pattern, or a similar substitute, QUIC graphs derives $\aform{\top}$.
\end{example}

\subsection{BDD-based Set Constraints}
\label{s:4:3:bdd}
% - BDD structure quickly explained
Binary decision diagrams (BDDs) are a canonical representation of Boolean algebraic functions.  There are three basic syntactic elements of a BDD.  The $\textsc{True}$ and $\textsc{False}$ elements represent the obvious constants, but $\textrm{ITE}(X,B_t,B_e)$ is an if-then-else structure.  If the variable $X$ is $\true$, the result of evaluating $B_t$ is returned, otherwise the result of evaluating $B_e$ is returned.
\begin{align*}
  B ::= \textsc{True} \ | \ \textsc{False} \ | \ \textrm{ITE}(X,B_t,B_e)
\end{align*}
What makes BDDs canonical is that we only consider reduced, ordered BDDs, where it is assumed that there is a total order $\prec$ on the variables.  An $\textrm{ITE}(X,B_t,B_e)$ can only be constructed if $X \prec X'$ for all variables $X'$ in $B_t$ or $B_e$.  Additionally, structural sharing is mandated, so the reuse of the same syntax is referentially identical to any other use of that syntax.

% - encoding of a class of set constraints (it gives the gamma)
The encoding of constraints maps operators from their constraint form (as in Definition~\ref{d:1:symsets}) to their Boolean algebraic form: $\cup \mapsto \vee, \ \cap \mapsto \wedge, \ \comp{} \mapsto \neg, \ \subseteq \mapsto \rightarrow, \ = \mapsto \leftrightarrow$. After singleton set constraints are dropped, the remaining constraints are directly and exactly represented by the BDD.  

% - algorithms main ideas
Domain operations are straightforward: $\adomjoin$ and $\adomwiden$ are implemented with the $\vee$ operation, which is precise and does not need any rules or heuristics; $\adomforget$ takes advantage of reasonably efficient quantifier elimination provided by BDDs and uses existential quantifier elimination to drop variables.  Queries such as $\adomisle$ are easily implemented using validity checking functionality provided by BDDs.  Critically, because BDDs are a canonical form, many operations such as $\adomforget$ and $\adomassume$ become much more efficient, whereas the operation $\adomisbot$ becomes an $O(1)$ check.

% - example
\begin{example}[BDD-based join]  Consider the same inputs as Example~\ref{ex:Qjoin}.  Encoding them to BDDs (and using some Boolean-algebraic notation as shorthand) yields the following results:
\begin{align*}
  & \astate_0 = \aform{X_1 \subseteq X_3 \logand X_3 \subseteq X_2} =
  \textrm{ITE}(X_1, X_2 \wedge X_3, \textrm{ITE}(X_2, \textsc{True},
  \neg X_3))
  \\
  & \astate_1 = \aform{X_1 \subseteq X_4 \logand X_4 \subseteq X_2} =
  \textrm{ITE}(X_1, X_2 \wedge X_4, \textrm{ITE}(X_2, \textsc{True},
  \neg X_4))
  \\
  & \adomjoin(\astate_0, \astate_1 ) =
  \textrm{ITE}(X_1, X_2 \wedge \textrm{ITE}(X_3, \textsc{True}, X_4),
  \\
  & \qquad \qquad \qquad \qquad \qquad \quad \;\;
  \textrm{ITE}(X_2, \textsc{True}, \textrm{ITE}(X_3, \neg X_4, \textsc{True})))
\end{align*}
  The result of this join is equivalent to the set constraints $\aform{X_1 \subseteq X_2}$, $\aform{X_1 \subseteq X_3 \cup X_4}$, and $\aform{X_3 \cap X_4 \subseteq X_2}$, which includes not only the obvious result of $\aform{X_1 \subseteq X_2}$, but also other, possibly useful results.  It is a precise join.
\end{example}

We implement the BDD abstraction on top of the CU decision diagrams package~\cite{cudd}, which is high performance and offers the ability to extract prime implicants (as in~\cite{prime:mit:92}).  The prime implicants of the negation of the Boolean function are easily converted to conjuncts of the form used by QUIC graphs.

\subsection{The Equalities Domain Functor: Compact Equality Constraints}
\label{s:4:4:eqs}
\newcommand{\eqrep}{\ensuremath{Q}}
% - origin of the equalities problem (cases where there are too many variables
%   and most of them are equal, consequences in terms of complexity)

When analyzing real programs, in addition to complex set constraints, there are often many very simple equality constraints of the form $\aform{X = Y}$.  These can be a problem in several ways.  In BDDs, for example, while equalities are normalized and handled precisely, they can grow the size of the BDDs significantly.  This results in significantly increased memory usage and decreased efficiency since many BDD operation rebuild the BDD.  In QUIC graphs, equalities grow the size of the graph, and place significantly more load on the pattern matching system, potentially causing a blow up in the number of constraints because QUIC graphs can represent each variant of an expression rewritten using all available equalities.  In linear set abstractions, there are similar potential problems.

As a result, abstractions like QUIC graphs and the linear set abstraction have special handling for equality.  This improves performance and precision at the cost of complexity.  Instead, much of this complexity can be moved outside the abstraction and handled by lifting the abstraction to one that keeps track of equalities separately from other kinds of constraints.

% - principle of the functor
The equality functor serves as an intermediary between the domain interface and the abstract domain that is being lifted.  It intercepts equality constraints and handles them externally preventing them from being seen by the underlying abstract domain.  This saves the domain from the cost and complexity of handling the equalities.

% - abstract states and concretization
The equality functor defines a set of equivalence classes $\eqrep$.  The set of equivalence classes is a map $\setvars \rightarrow \setvars$ that maps each variable to the chosen representative for the equivalence class.  The functor then lifts an abstract state $\adom$ into a tuple $(\adom, \eqrep)$.  In the lifting, $\adom$ is restricted to only have symbols that are representatives for the equivalence class.  Therefore, when an equality is added that merges two equivalence classes, the resulting representative replaces the two previous representatives in $\adom$.

The concretization ensures that all symbols in the same equivalence class map to the same concrete set:
\begin{align*}
\gamma((\eqrep,\adom)) = \{ \ \state \ | \ \state \in \gamma(\adom) \wedge \forall X,Y \in \setvars^2. \; \eqrep(X) = \eqrep(Y) \rightarrow \state(X) = \state(Y) \ \}
\end{align*}

Domain operations $\adomjoin$, $\adomwiden$, and $\adomisle$, unify their corresponding $\eqrep$s pushing any non-common equalities into the underlying domain.  This ensure that the underlying domain determines the precision, but it is not required to handle most of the load of the equalities.  The $\adomassume$ operation rewrites the constraint, extracting the equalities and rewriting remaining variables to their representatives before passing the constraint to the underlying domain.
 

% - example
\begin{example}[Equality functor join]
Consider the following two abstract states, where the underlying domain is just shown as symbolic set constraints:
\begin{align*}
  \astate_0 = ([W \mapsto W, X \mapsto W, Y \mapsto W], \aform{W \subseteq Z}) \ \ 
  \astate_1 = ([X \mapsto X, Y \mapsto X], \aform{W \subseteq X \wedge X \subseteq Z})
\end{align*}
In the join, the equivalence classes are unified, producing the resulting $\eqrep$: $[X \mapsto X, Y \mapsto X]$.  The equality $\aform{W = X}$ from $\astate_0$ is not represented in the unification, so it is added back to the underlying domain in $\astate_0$.  The result is therefore
\begin{align*}
  ([X \mapsto X, Y \mapsto X], \adomjoin(\aform{W = X \wedge W \subseteq Z}, \aform{W \subseteq X \wedge X \subseteq Z}))
\end{align*}
\end{example}

\subsection{The Packing Domain Functor: Sparse Constraints}
\label{s:4:5:packs}
% - issue with relational abstract domains
Most relational domains have a complexity that is related to the number of variables constrained by the abstract state.  For example, BDDs, in the worst case, are exponential in the number of variables.  However, in many programs, there are relatively small clusters of variables that are related.  Therefore it is possible to increase the efficiency of analysis by representing each cluster of variables by a separate abstract state~\cite{ens:pldi:03}.

% - packing principle (citations to Astree)
If each of $m$ clusters of $n$ variables is represented by a separate abstract state, rather than operations having a complexity of, for example, $O(2^{m\cdot n})$, they can have complexity $O(m\cdot 2^n)$.  To do this, initially all variables are assumed to be in their own cluster.  Clusters are merged whenever variables from each cluster occur in the same constraint.  In this way the clusters are dynamically determined, which is required when an abstract domain is used as a library and thus a pre-analysis cannot be performed.

% - abstract states and concretization
An abstract state in the cluster functor consists of one of three values: $\top$, $\bot$, or a map $M : \#_M \rightarrow \adom$ that maps cluster ids in $\#_M$ to abstract states from the domain being lifted.  The $\top$ and $\bot$ values concretize as they do in Figure~\ref{f:4:symsets}.  The map concretizes as follows:
\begin{align*}
  \gamma(M) = \{ \state \ | \ \forall \astate \in \textrm{Range}(M). \; \state \in \gamma(\astate) \}
\end{align*}

% - example
\begin{example}[Constraining a packed abstract state]
Consider the following abstract state represented by the logic from Definition~\ref{d:1:symsets}, lifted into two packs with ids $0$, and $1$:
$
  \astate_0 = [0 \mapsto \aform{X_0 \subseteq X_1}; \; 1 \mapsto \aform{Y_0 \subseteq Y_1} ]
$.
The operation $\adomassume(\astate_0, Y_1 \subseteq Y_2)$ operates only on pack id $1$.  It does not have to involve any computation on pack $0$.  The resulting pack $1$ is: $1 \mapsto \adomassume(\aform{Y_0 \subseteq Y_1}, Y_1 \subseteq Y_2)$.
\end{example}


%5. Solver-based set abstractions (1.5 pages)
%-> encode problem to traditional, known problem
%5.1 SMT
%5.2 QBF
\section{Solver-based Abstractions}
\label{sec:solvers}

%6. Evaluation (2 pages)
%-> compare approaches in several classes of problems
%- Benchmark suites:
%- Python set tests
%- JSAna JavaScript verification
%- Memcad C data structure memory safety
%- Show:
%- Old set domains much slower
%- New set domains faster and often more precise
%- SMT doesn't scale in on these applications
\section{Evaluation}
\label{sec:evaluation}

In this section, we evaluate the set abstractions.  We aim to answer the following questions about set abstractions. Can set abstractions be sufficiently precise to be useful? Can precision be made available while providing scalability? What trade-offs are necessary to achieve scalability?  To evaluate these questions, we look at three different applications of set abstractions: (1) The expressible subset of tests of the Python set data structure (as used for QUIC graphs~\cite{ab:ecoop:13}); (2) Traces of set domain operations as used in JSAna to verify functions in selected JavaScript libraries (from \cite{hoo:14:sas,desync:15:esop}); and (3) Traces of set domain operations as used in Memcad to perform shape analysis in the presence of unstructured sharing (from \cite{memcad:15:sas}).  Results are shown in Table~\ref{tab:results}.

\begin{table}[t]
    \caption{Number of proved properties ($\adomprove$), average aggregate run time for non-timed-out benchmarks (Time), and number of timed-out benchmarks (TO) for 24 Memcad benchmarks, 5 JSAna benchmarks, and 24 Python benchmarks.}
    \label{tab:results}
    \begin{tabularx}{\textwidth}{l*{3}{Xr@{/}lS[table-format=2.3, round-mode=places, round-precision=3]@{}>{(}l<{)}}}
\toprule
Config & & \multicolumn{4}{c}{Memcad(24)} & & \multicolumn{4}{c}{JSAna(5)} & & \multicolumn{4}{c}{Python(24)} \\
       & & \multicolumn{2}{c}{$\adomprove$} & \multicolumn{2}{c}{Time(TO)}
       & & \multicolumn{2}{c}{$\adomprove$} & \multicolumn{2}{c}{Time(TO)}
       & & \multicolumn{2}{c}{$\adomprove$} & \multicolumn{2}{c}{Time(TO)} \\
\midrule
% lin
\textbf{lin}   & & 612 & 1366 & 0.035967349969752  &  0 & &   0 & 525 & 0.434864892014519 & 0 & &  4 & 42 & 0.00370062068965517 & 0 \\
       eq      & & 608 & 1366 & 0.0353923793103448 &  0 & &   0 & 525 & 0.23451724984876  & 0 & &  4 & 42 & 0.00706779310344828 & 0 \\
       pack    & & 612 & 1366 & 0.0490982722323049 &  0 & &   0 & 525 & 0.651664516333938 & 0 & &  4 & 42 & 0.00648337931034483 & 0 \\
       eq+pack & & 609 & 1366 & 0.0445209485783424 &  0 & &   0 & 525 & 0.784750289776164 & 0 & &  4 & 42 & 0.0105770862068966  & 0 \\
       pack+eq & & 608 & 1366 & 0.0672156947973382 &  0 & &   0 & 525 & 0.393046004839685 & 0 & &  4 & 42 & 0.0109433965517241  & 0 \\
\midrule
% bdd
\textbf{bdd}   & & 612 & 1366 & 0.0207534445919044 &  0 & & 176 & 525 & 21.7929897051295  & 0 & & 34 & 42 & 0.105308375580624   & 0 \\
       eq      & & 612 & 1366 & 0.0410980656934307 &  0 & & 176 & 525 & 1.2059448669172   & 0 & & 34 & 42 & 0.112157598540146   & 0 \\
       pack    & & 612 & 1366 & 0.0515431181154612 &  0 & & 176 & 525 & 0.261539623449577 & 0 & & 34 & 42 & 0.109233451891175   & 0 \\
       eq+pack & & 612 & 1366 & 0.0549087763769078 &  0 & & 176 & 525 & 1.69208972207256  & 0 & & 34 & 42 & 0.11571100530856    & 0 \\
       pack+eq & & 612 & 1366 & 0.0861057770404778 &  0 & & 176 & 525 & 1.79561763395081  & 0 & & 34 & 42 & 0.118790341738553   & 0 \\
\midrule
% quic
\textbf{quic}  & & 596 & 1366 & 4.29938308257713   &  1 & & 155 & 525 & 54.6163486718088  & 0 & & 20 & 39 & 0.412252620689655   & 2 \\
       eq      & & 549 & 1366 & 2.2887479431337    &  0 & & 116 & 525 & 4.63347001754386  & 0 & & 18 & 39 & 0.416192155172414   & 2 \\
       pack    & & 605 & 1366 & 5.55630652843315   &  0 & & 155 & 525 & 48.5171079585602  & 0 & & 20 & 39 & 0.454328344827586   & 2 \\
       eq+pack & & 549 & 1366 & 2.30702902782819   &  0 & & 121 & 525 & 8.20119367967332  & 0 & & 18 & 39 & 0.455532620689655   & 2 \\
       pack+eq & &  55 &   58 & 0.0797614395039322 & 10 & & 121 & 525 & 9.30715310889292  & 0 & & 17 & 38 & 0.362135120689655   & 3 \\
\midrule
% smt
\textbf{smt}   & & 177 &  315 & 44.9951502434967   &  4 & &  12 &  23 & 0.038717701754386 & 4 & & 34 & 41 & 0.296424896551724   & 1 \\
       eq      & & 416 &  927 & 35.7984346485178   &  1 & &  62 & 152 & 5.75257775045372  & 2 & & 31 & 41 & 0.293945672413793   & 1 \\
       pack    & & 177 &  315 & 16.329403277072    &  4 & &  12 &  23 & 0.786895052631579 & 4 & & 34 & 41 & 5.55303881034483    & 1 \\
       eq+pack & & 438 &  927 & 40.6210611098004   &  1 & &  27 &  73 & 9.35488968421053  & 3 & & 31 & 41 & 10.8835764137931    & 1 \\
       pack+eq & & 231 &  458 & 12.0267939812462   &  3 & &  12 &  23 & 0.608693561403509 & 4 & & 31 & 41 & 10.8377189137931    & 1 \\
\bottomrule
\end{tabularx} 

\end{table}

Because the definition of necessary precision depends on the use of a domain, we measure precision by comparing against a standard for precision.  For Memcad, the linear set abstraction (\textbf{lin}) was designed to be as precise as is needed for the Memcad benchmarks.  This means that any abstraction that achieves the same number of proofs without timeout is sufficiently precise.  It is important to note that many of these proofs are not intended to succeed.  They are used as queries internally with the analysis, so it is not possible to achieve 100\%.  From the results, we see that all of the BDD-based abstractions (\textbf{bdd}) achieve this.  We can also see that the equality (eq) and packing (pack) functor, regardless of the order in which they are applied do not change precision when applied to the BDD.  However, when applied to the linear set abstraction, they do sometime change precision.  This is because they affect internal representation and potentially affect the heuristics used within the abstract domain.  QUIC graphs (\textbf{quic}) and SMT (\textbf{smt}) do not perform as well under any configuration.  The reason for this is that QUIC graphs does not employ appropriate heuristics for all of the cases needed by Memcad and both have performance problems that cause them to time out before completing some benchmarks.

For the JSAna benchmarks, the BDD abstraction was designed to meet its precision needs and adding the equality or packing functor do not affect precision in any way.  It only affects performance.  However, the linear sets abstraction is not able to cope with the non-disjoint-union constraints that arise frequently in the JSAna benchmarks and thus loses all precision rapidly.  By comparison, QUIC graphs performs well.  It is unable to prove as many properties as JSAna, but it is still able to prove many properties.  Once again, tuning the heuristics could improve this precision, but possibly at the cost of performance.  SMT, once again does not perform well because of performance problems.  On the benchmarks where it completes, it is identical in precision to BDDs.

The Python benchmarks are slightly different because they are an analysis with just a few properties.  Each of these properties is something to verify, so the target is 100\%.  Here we see that none of the abstractions are able to achieve 100\%.  The linear set abstraction cannot achieve this because it is unable to represent the non-disjoint-union constructs.  The BDD and SMT abstractions cannot achieve 100\% because they do not support full cardinality reasoning.  Once again, QUIC graphs is insufficient because of the limited heuristics it employs as well as some performance problems.

The scalability of the abstractions can be seen in Table~\ref{tab:results} in the total analysis time, which measures the time to run the full benchmark suite, on average.  The times are only directly comparable if there are no time outs, which happens after 60 seconds per benchmark.  We first see that the linear domain is reliably fast.  Applying the equality and packing functors generally does not affect performance significantly.  BDDs, by comparison are less reliable.  While in the Memcad benchmarks, they perform well, nearly matching the linear domain, in the JSAna benchmarks we see significant variability.  In fact, without any of the functors, performance can be unacceptably slow at almost 22 seconds to analyze five functions.  However, the addition of, in particular, the packing functor, makes a significant difference.  It lowers the cost of the analysis, without losing any precision down to a fraction of a second.  However, the variability here indicates that depending on the particular benchmark (or, in fact, the BDD implementation), the optimum combination of functors may vary.  However, selecting the packing functor seems to be a benefit without significant risk.  The QUIC graphs performance is unreliable.  Due to the expensive pattern matching machinery, it does not compare in terms of performance, though it is helped significantly by the equality functor, at the cost of precision.  The SMT domain fails to perform timing out on at least one test in each benchmark suite.  This is because the SMT solver is failing to perform incrementally.  In essence, it has the same workload as the BDD, but it performs these proofs lazily.  This laziness is not necessarily a problem if work can be reused from one proof to the next, but it appears that this is not the case right now.  We suspect that the combination of doing validity proofs (instead of satisfiability queries) with quantifiers is preventing this reuse.

The results make clear four things.  First, if it is possible to design a targeted abstraction as the linear abstraction is for Memcad, it is worth it.  The performance is reliable and the precision is predictable.  Second, if it is not clear what the constraints may be, BDDs provide a good alternative that gives excellent (if not perfect due to the insufficient cardinality reasoning) precision with the risk of unreliable performance.  Third, much of the risk can be eliminated through the use of functors.  For equality heavy loads, the equality functor provides a significant benefit.  The packing functor seems to reliably improve performance by simply lowering the cost of each BDD operation without any measurable impact on precision.  Lastly, unless the content-centric reasoning of QUIC graphs is necessary, it does not make sense to use it due to both unreliable performance and precision.  Similarly, with the current state of SMT, this is not an appropriate use.  It may be possible to fix this, but today, it remains impractical for performance reasons.


% want to show that set abstractions can scale.



% want to show expert system for deciding on set abstractions

% want to show that SMT has performance difficulties due to the laziness of the abstraction

%7. Trade-offs/Limitations (0.5 pages)
%-> limitations and possible enhancements
%- Cardinality
%- Contents
%
%8. Related Work (1 page)
\section{Conclusions and Related Work}
The problem of creating scalable, precise, and predictable abstractions for sets remains challenging.  This paper introduced several ways of approaching this problem and showed that for symbolic set abstractions, binary decision diagrams offer good performance, precision, and predictability trade-offs.  However, it is preferable to craft a simplified custom abstraction such as the linear abstraction.  This offers more predictable performance in exchange for targeted precision rather than general precision.

There are other set abstractions available.  They all offer different functionality at different costs.  The QUIC graphs abstraction~\cite{ab:ecoop:13,quicr:cav:14} focuses on combining reasoning about contents with symbolic set reasoning.  This comes at the cost of performance, precision, and predictability when it comes to purely symbolic set reasoning.  The FixBag abstraction~\cite{fixbag:cav:11} attacks the problems of multisets or bags offering cardinality reasoning as well as symbolic set reasoning.  Similar to QUIC graphs, it exchanges performance, precision, and predictability for this functionality.  The linear and the BDD-based abstractions we present here are simple, providing little support for cardinality in exchange for performance, precision, and predictability.

There are several decision procedures for sets.  Bradley et al.~\cite{bradley:vmcai:06} introduced a decision procedure for set contents and relationships (without cardinality).  BAPA~\cite{knr:jar:06,jahob:thesis:07} is a decision procedure for sets with cardinality.  Z3~\cite{mb:tacas:08} also includes a decision procedure for sets with contents.  None of these decision procedures are designed for invariant generation.  It is possible that interpolation procedures~\cite{interp:cav:03} could be designed based upon these procedures, but to our knowledge this has not been done.  Regardless, without invariant generation that is compatible with static analysis, it is difficult to use software as a component of an existing analysis.

Due to the prevalence of Boolean algebra in the algorithms presented here, there is a natural correspondence to hardware model checking~\cite{mc:toplas:86}.  There are several differences, however.  First, the algorithms described here are designed to be integrated with a static analyzer, whereas a hardware model checkers are not.  The big difference is that hardware model checkers combine the control flow with the data flow into one single logic system.  Here, we maintain the difference as the host static analyzer dictates how the abstract domain is used.  This means that abstract domains can only see the constraints and domain operations as they occur, whereas a hardware model checker can see the whole analysis globally.  As a result, it is difficult to apply hardware model checkers in the environments where it is possible to apply abstract domains.

As a result, we find that for now, abstractions that construct normal forms, such as the linear abstraction and binary decision diagrams offer the best way of handling sets in static analysis.  We have shown that depending on the application, both of these techniques offer sufficient performance and precision, especially when combined with functors for performing packing and managing equalities.  The end result is that these abstractions are scalable, precise, and predictable in their behavior.
%
%9. Conclusions (0.5 pages)

\subsubsection{Acknowledgements.} This material is based upon work supported in
part by a Chateaubriand Fellowship, by the National Science Foundation under
Grant Numbers CCF-1055066 and CCF-1218208, and by the European Research Council
under the FP7 grant agreement 278673 (Project MemCAD).
    
%----------------------------------------------------------------------------
% Biblio.

\bibliographystyle{abbrv}
\bibliography{symbolic-sets}
\end{document}
