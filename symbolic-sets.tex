\documentclass{llncs}
\usepackage{amsmath,amssymb}
\usepackage{tikz}
\usetikzlibrary{matrix,chains,scopes}

\title{Abstractions for Symbolic Sets}

\author{Arlen Cox, Xavier Rival}

\institute{Inria/CNRS/ENS Paris}

\begin{document}
    
\maketitle
%-----------------------------------------%
% Colors                                  %
%-----------------------------------------%
\newrgbcolor{mblue}{        0.1   0.1   0.5 }
\newrgbcolor{mbrown}{       0.57  0.4   0.3 }
\newrgbcolor{mcyan}{        0.1   0.1   0.5 }
\newrgbcolor{mgreen}{       0.2   0.5   0.2 }
\newrgbcolor{mmidgray}{     0.65  0.65  0.65}
\newrgbcolor{morange}{      0.8   0.45  0.45}
\newrgbcolor{mpurple}{      0.5   0.1   0.5 }
\newrgbcolor{mred}{         1     0     0   }
\newrgbcolor{mhigray}{      0.21  0.21  0.21}
\newrgbcolor{mgray}{        0.5   0.5   0.5 }
\newrgbcolor{mmidgreen}{    0.3   0.8   0.4 }
\newrgbcolor{mlightgray}{   0.85  0.85  0.85}
\newrgbcolor{mlightblue}{   0.7   0.85  1.00}
\newrgbcolor{mlightgreen}{  0.7   1.00  0.85}
\newrgbcolor{mlightpurple}{ 1.00  0.7   1.00}
\newrgbcolor{mlightred} {   1.00  0.7   0.7 }
\newrgbcolor{mlightyellow}{ 1.00  1.00  0.6 }
\newrgbcolor{mulightgray}{  0.98  0.98  0.98}
\newrgbcolor{mulightblue}{  0.9   0.95  1.00}
\newrgbcolor{mulightgreen}{ 0.9   1.00  0.95}
\newrgbcolor{mulightpurple}{1.00  0.9   1.00}
\newrgbcolor{mulightred} {  1.00  0.85  0.85}
\newrgbcolor{mulightyellow}{1.00  1.00  0.87}
\newrgbcolor{mvlightblue}{  0.8   0.9   1.00}
\newrgbcolor{mvlightgreen}{ 0.8   1.00  0.9 }
\newrgbcolor{mvlightpurple}{1.00  0.8   1.00}
\newrgbcolor{mvlightred} {  1.00  0.85  0.85}
\newrgbcolor{mvlightyellow}{1.00  1.00  0.73}
\newrgbcolor{myellow}{      1.00  1.00  0.30}
\newrgbcolor{mwhite}{       0.99  0.99  0.99}

%-----------------------------------------%
% Typing and abbreviations                %
%-----------------------------------------%
% Standard Abbrevs
\newcommand{\eg}{e.g.\xspace}
\newcommand{\ie}{i.e.\xspace}
\newcommand{\resp}{resp.\xspace}
% Removing
\newcommand{\commentout}[1]{}
% Project names
\newcommand{\memcad}{{MemCAD}\xspace}

%-----------------------------------------%
% Math standard stuffs                    %
%-----------------------------------------%
% Abstraction
\newcommand{\abs}[1]{{#1}^{\sharp}}
% Booleans
\newcommand{\true}{\mathbf{true}}
\newcommand{\false}{\mathbf{false}}
% Definitions
\newcommand{\isdef}{::=}
% Logical relations
\newcommand{\logor}{\mathrel{\vee}}
\newcommand{\logand}{\mathrel{\wedge}}
\newcommand{\suplus}{\mathrel{\uplus}}
% Powerset
\newcommand{\partsof}[1]{\mcP(#1)}
% Semantics
\newcommand{\sem}[1]{\llbracket #1 \rrbracket}
\newcommand{\asem}[1]{\abs{\llbracket #1 \rrbracket}}

%-----------------------------------------%
% Inserting Tikz pictures                 %
%-----------------------------------------%
% Including pictures
\newcommand{\tikzpics}[2]{\scalebox{#1}{\input{tkz-#2.tex}}} % with scale
\newcommand{\tikzpic}[1]{\tikzpics{1}{#1}}                   % without scale
% Putting a formula
\newcommand{\tkzputform}[2]{\node[] at (#1){\ensuremath{#2}};}
\newcommand{\tkzputrghform}[2]{
  \node[label distance=-10pt,inner sep=0pt,outer sep=0pt,
    minimum height=0pt,minimum width=0pt,
    outer xsep=0pt,outer ysep=0pt,inner xsep=0pt,inner ysep=0pt,
    label=right:\ensuremath{#2}] at (#1){};}
\newcommand{\tkzputlftform}[2]{
  \node[label distance=-10pt,inner sep=0pt,outer sep=0pt,
    minimum height=0pt,minimum width=0pt,
    outer xsep=0pt,outer ysep=0pt,inner xsep=0pt,inner ysep=0pt,
    label=left:\ensuremath{#2}] at (#1){};}
% Putting a piece of text
\newcommand{\tkzputtext}[2]{\node[] at (#1){#2};}
\newcommand{\tkzputrghtext}[2]{
  \node[label distance=-10pt,inner sep=0pt,outer sep=0pt,
    minimum height=0pt,minimum width=0pt,
    outer xsep=0pt,outer ysep=0pt,inner xsep=0pt,inner ysep=0pt,
    label=right:#2] at (#1){};}
\newcommand{\tkzputlfttext}[2]{
  \node[label distance=-10pt,inner sep=0pt,outer sep=0pt,
    minimum height=0pt,minimum width=0pt,
    outer xsep=0pt,outer ysep=0pt,inner xsep=0pt,inner ysep=0pt,
    label=left:#2] at (#1){};}
% Pointers
\newcommand{\tkptrdot}[1]{\draw (#1) [fill=black] circle (0.05);}

%-----------------------------------------%
% Programs                                %
%-----------------------------------------%
% Variables
\newcommand{\ttvar}[1]{\texttt{#1}}
\newcommand{\cvar}[1]{\ensuremath{\texttt{#1}}}
\newcommand{\varc}{\ttvar{c}}
\newcommand{\vard}{\ttvar{d}}
\newcommand{\varg}{\ttvar{g}}
\newcommand{\varh}{\ttvar{h}}
\newcommand{\vari}{\ttvar{i}}
\newcommand{\varl}{\ttvar{l}}
\newcommand{\varm}{\ttvar{m}}
\newcommand{\varn}{\ttvar{n}}
\newcommand{\varp}{\ttvar{p}}
\newcommand{\varr}{\ttvar{r}}
\newcommand{\vars}{\ttvar{s}}
\newcommand{\vart}{\ttvar{t}}
\newcommand{\varu}{\ttvar{u}}
\newcommand{\varv}{\ttvar{v}}
\newcommand{\varx}{\ttvar{x}}
\newcommand{\vary}{\ttvar{y}}
% Pointers
\newcommand{\nullptr}{\textbf{0x0}}

%-----------------------------------------%
% Examples                                %
%-----------------------------------------%
\newcommand{\naddr}[1]{\texttt{n}_{#1}} % Node addresses

%-----------------------------------------%
% Set abstraction                         %
%-----------------------------------------%
% set of all set variables
\newcommand{\setvars}{\bbX_{\bf s}}
% set variables (in case we change font)
\newcommand{\setvw}{W}
\newcommand{\setvx}{X}
\newcommand{\setvy}{Y}
\newcommand{\setvz}{Z}
% set of all elements
\newcommand{\values}{\bbV}
% set of concrete states
\newcommand{\states}{\bbS}
% symbolic sets
\newcommand{\symsets}{\bbC}
% abstract domain (generic)
\newcommand{\adom}{\abs{\bbD}}     % abstract domain
\newcommand{\astate}{\abs{\state}} % generic abstrct element
\newcommand{\gammadom}{\gamma}     % concretization function
% extremal elements
\newcommand{\adombot}{\bot_{\adom}}
\newcommand{\adomtop}{\top_{\adom}}
% abstract operations
\newcommand{\adomisbot}{\textbf{isbot}_{\adom}}
\newcommand{\adomforget}{\textbf{forget}_{\adom}}
\newcommand{\adomassume}{\textbf{assume}_{\adom}}
\newcommand{\adomprove}{\textbf{prove}_{\adom}}
\newcommand{\adomjoin}{\textbf{join}_{\adom}}
\newcommand{\adomwiden}{\textbf{widen}_{\adom}}
\newcommand{\adomisle}{\textbf{is\_le}_{\adom}}
\newcommand{\adommeet}{\textbf{meet}_{\adom}}

%-----------------------------------------%
% single abstract state
\renewcommand{\abstract}{\Sigma}
% set of all abstract states
\newcommand{\abstracts}{\textsc{AbsStates}}
% cardinality of a set
\newcommand{\card}[1]{|{#1}|}
% complement of a set
\newcommand{\comp}[1]{{#1}^{\mathrm{c}}}
% individual set variables: W, X, Y, or Z
% set of all set variables
%\newcommand{\setvars}{\textsc{SetVars}}
% concrete state
\newcommand{\state}{\sigma}
% set of all concrete states
%\newcommand{\states}{\textsc{States}}
% set of all values
%\newcommand{\values}{\textsc{Vals}}
% concrete set c
% syntax of a logical constraint L
% syntax of a set expression E
% power set of a set
\newcommand{\powerset}[1]{\mathcal{P}\left({#1}\right)}


\newcommand{\defeq}{\stackrel{\textrm{\tiny def}}{=}}

%1. Introduction (1.5 pages)
%- Goals:
%- Set abstractions serve a different purpose than SAT/SMT/MC
%- Inference vs Entailment
%- Explicit vs Implicit control flow
%- Handling of quantifiers
%- Forward vs backward analysis
%- Modern set abstractions offer performance and precision
%- Set abstractions are useful for a wide variety of problems
\section{Introduction}
\label{s:1:intro}
The verification of program properties that
involve data structures is a challenging problem~\cite{jahob:thesis:07,compass:popl:11,fixbag:cav:11,celia:vmcai:12,ab:ecoop:13,hoo:14:sas,memcad:15:sas}.
One key reason for this is that
if a data structure is unbounded, there is a potentially unbounded number of constraints on its elements.  Since these constraints often affect important properties such as memory safety~\cite{memcad:15:sas}, functional correctness~\cite{fixbag:cav:11}, or basic program behavior~\cite{hoo:14:sas}, it is vital to develop techniques for efficiently reasoning about relationships between unbounded numbers of elements.

This paper focuses on the use of set constraints to reason about unbounded collections of elements.  Set constraints can be used to dynamically partition data structures, correlate collections of elements with one another, or determine analysis case splits.  They are useful for representing data and pointers relationships in structures such as maps, graphs, lists, sets, and arrays.  They can be combined with other techniques such as separation logic~\cite{hoo:14:sas,memcad:15:sas} and numerical analyses~\cite{quicr:cav:14} to enhance those analyses.

%% xr: need to adjust overview, there is some repetition now
For example, consider the program in Figure~\ref{fig:intro-example}
that copies one map on top of another.
Within the loop, there is a complex relationship between the sets of
keys of \ttvar{src} and \ttvar{dst}.
At the specified point, the keys of \ttvar{src} can be partitioned into
three parts.
The keys already visited $X_v$ by the loop, the element currently being
visited by the loop $\{\varx\}$, and the keys not visited $X_n$ by
the loop.
The keys of \ttvar{dst} can be partitioned into those originally in
\ttvar{dst} that have not been overwritten, and those $X_v$ that have
been overwritten or added from \ttvar{src}.
This set reasoning allows precise symbolic tracking of the provenance
of map partitions.

\begin{figure}[tb]
  \newbox\exprogbox
  \begin{lrbox}{\exprogbox}
    \begin{minipage}[t][1cm][b]{0.4\textwidth}
      \begin{lstlisting}[language=python]
def extend(dst, src):
  for x in src:
    # invariant point
    dst[x] = src[x]
      \end{lstlisting}
    \end{minipage}
  \end{lrbox}
  \newbox\exproginv
  \begin{lrbox}{\exproginv}
    \begin{minipage}[t][0.9cm][b]{0.4\textwidth}
      \begin{align*}
        \exists X_v, X_n. \; keys(\ttvar{src})
        & = X_v \suplus \{\varx\} \suplus X_n
        \\
        {} \logand keys(\ttvar{dst})
        & = (keys(\ttvar{dst})_0 \setminus X_v) \suplus X_v
      \end{align*}
    \end{minipage}
  \end{lrbox}
  \centering
  \subfigure[Function to copy all keys and values from map \ttvar{src}
  onto map \ttvar{dst}]{\usebox\exprogbox}
  \quad
  \subfigure[Set abstraction at \texttt{\textit{invariant point}}]{%
    \usebox\exproginv}
  \caption{Set constraints can relate portions of data structures}
  \label{fig:intro-example} \label{f:1:intro}
\end{figure}

This paper focuses on abstractions for states described by the logic for \emph{symbolic sets}.  The logic consists of a Boolean algebra over the set variables with singleton sets. We find that this subset is sufficiently large to be useful and we
believe that it serves as a good starting point for extensions to the
logic, such as reasoning about explicit set contents or more precise
cardinality.

However, despite the fact that we are not reasoning about the values
contained in sets or complex cardinalities, Boolean algebras, by
themselves, are challenging for invariant generation.
Naive approaches such as saturation and pattern matching rarely work
without complex heuristics~\cite{fixbag:cav:11,ab:ecoop:13}.
It is unavoidable that the worst-case time for precise invariant
generation will be exponential because of the Boolean algebra.
However, it is desirable that invariant generation should be efficient
in the common cases, and unlike systems that involve complex heuristics,
lose precision only in understandable and predictable ways.

% AC: seems like this could be tightened up and combined with the next paragraph
In this paper we aim to design scalable, precise, and predictable abstractions for symbolic sets by combining new abstract domains with performance/precision-enhancing functors that lift existing set abstractions to new set abstractions.  Specifically, we make the following contributions:
%This includes abstractions based on binary decision diagrams,
%satisfiability modulo theories, and linear set constraints.
%Additionally, it includes performance and precision enhancing combinators for tracking
%singleton sets, handling equality, and doing dynamic variable packing.
%These abstractions and combinators exist within a general framework so
%that additional abstractions and combinators can be easily added.
%This framework is available within the QUICr library, which is now
%used by two research analyzers.
%
%In this paper we make the following contributions.
\begin{compactitem}
\item We define a common interface for symbolic set abstractions that is designed to meet the needs of static analyzers (Section~\ref{sec:logic-and-set-abstraction}).
%
\item Using specialized data structures, we construct a battery of symbolic set abstract domains and performance/precision-enhancing functors designed to target real-world data structure verification problems (Section~\ref{sec:constructed}).
%
\item We adapt an off-the-shelf satisfiability-modulo-theories solver to the set abstraction interface (Section~\ref{sec:solvers}).
%
\item We compare abstractions for symbolic sets, finding that, while specialized abstractions are preferable, binary decision diagrams lifted with dynamic packing is a good compromise in scalability, performance, and predictability (Section~\ref{sec:evaluation}).
%
\end{compactitem}

%2. Overview (2 pages)
%-> a couple of examples, showing very intuitively (possibly just with pictures)
%the need for set abstractions
%Possibly:
%- a quick picture of HOO
%- a quick picture of Huisong's work
\section{Overview}
\label{s:2:over}
In this section, we present two static analyses that make use of set
reasoning in order to compute high level semantic properties of
programs.
These analyses rely on abstract interpretation~\cite{cc:popl:77} and
on an abstraction of program states, that describes data structures
and their contents.
An abstract domain defines a set of predicates that an analysis may
use, as well as operators to over-approximate the effect of program
behaviors on these predicates, and their implementation.

\paragraph{Inferrence of properties of open objects.}
Dynamic programming languages such as JavaScript feature {\em open objects}
that support dynamic addition and deletion of attributes and iteration over
them.
The analysis presented in~\cite{hoo:14:sas} computes relations
among objects, so as to verify programs as the piece of code of
Figure~\ref{f:1:intro}.
To achieve this, it infers relations between the sets of attributes
of distinct objects.
Objects have an unbounded number of attributes, thus the analysis requires
some abstraction over the attributes and their contents.
\newcommand{\varsrc}{\cvar{src}}
\newcommand{\vardst}{\cvar{dst}}
\begin{figure}[t]
  \newcommand{\picscale}{0.82}
  \begin{center}
    \tikzpics{\picscale}{hoo-inv}
  \end{center}
  \caption{Open objects and their abstraction}
  \label{f:2:hoo}
\end{figure}
Figure~\ref{f:2:hoo} represents a very simplified state, at the loop
head and after two iteration (thus two fields were copied).
We focus on the set of attributes of each object, and ignore their
contents (which could be described using similar techniques).
To abstract precisely the relations between the attributes of both objects
(\ie, in this case, copied attributes are common to both objects), we need
to describe the fields of each object as the union of a series of sets of
attributes, and to express relations among these sets.
The first purpose of the set abstract domain is to represent such set
relations.
The right hand side of Figure~\ref{f:2:hoo} depicts such an abstract
state, where \( X_n, X_r, X_v \) stand for sets of attributes, which are
made explicit, in the case of the left hand side concrete state.

Moreover, to infer these invariants, the analysis needs to reason
both about object structures and about attributes sets:
\begin{asparaitem}
\item Initially, no set relation should be assumed, and the fields of
  each object should be associated to a set of attributes;
\item When the analysis enters the body of the loop, it needs to
  {\em single out} attribute \( \mathtt{x} \), thus, to replace set
  \( X_v \) by \( X_v \suplus \{ \mathtt{x} \} \) (which produces the
  equalities of Figure~\ref{f:2:hoo});
\item When it exits the loop, the analysis should {\em generalize}
  both the object and set constraints abstractions, which requires
  to {\em eliminate} singleton \( \{ \mathtt{x} \} \) from the
  equations (it is visible only in the loop body) and to synthesize
  a new, more general collection of constraints.
\end{asparaitem}
To allow these steps, the set abstraction should provide basic operations
over set predicates, including (1) the addition of a set constraint, (2)
the proving of a set constraint, (3) the removal of a set variable and
(4) the generalization of two set abstract states.

\paragraph{Shape analysis in presence of unstructured sharing.}
The shape analysis for data-structures with unbounded sharing presented
in~\cite{memcad:15:sas} relies on separation logic~\cite{r:lics:02} to
describe memory states, and on inductive definitions to summarize
unbounded structures such as lists.
Unstructured sharing is very challenging as it cannot be described using
conventional inductive definitions.
Figure~\ref{f:3:memcad} displays the representation of a three nodes
graph using an adjacency list data-structure in the left.
To summarize such a structure using inductive predicates in separation
logic, \cite{memcad:15:sas} proposes to augment the list inductive
predicates with set information, which express where edges may point to.
This representation is shown in the right of Figure~\ref{f:3:memcad} in
a form where the first node is kept materialized.
It both asserts that edges of that node as well as edges from other nodes
point to the address of a valid node, namely an element of \( \{ \naddr{0}
\} \suplus \mcE \).
\begin{figure}[t]
  \newcommand{\picscale}{0.9}
  \tikzpics{\picscale}{memcad-inv}
  \caption{Summarization of an adjacency list-based graph representation}
  \label{f:3:memcad}
\end{figure}
The analysis of~\cite{memcad:15:sas} introduces a summary predicate
\( \textbf{graph}( \naddr{0}, \mcN ) \) where \( \naddr{0} \) is the
address of the first node and \( \mcN \) the set of all node addresses.
This predicate is defined by induction over the ``backbone'' of the
structure, and fully takes into account the property that all
\( \fldnext \) edges point to a valid node address in \( \naddr{0} \).
Henceforth, abstract states comprise both a {\em memory} part (which
consists of a formlula in separation logic with inductive predicates)
and a {\em set abstraction}.

To compute such summaries, the analysis needs to perform similar
operations as the analysis for open objects, in order to add set
constraints to the set abstract state, prove set constraints,
remove set variables and generalize abstract states.

% \paragraph{Need for set reasoning.}
% Both analyses consist of a composite abstract domain, that comprises a
% {\em structural abstract domain} describing the shape of structures, and
% of a {\em set abstract domain} expressing relations among set and value
% (pointers, field names, contents...) entities.


%3. Set Abstraction Problem (1 page)
%-> set up problem framework / why abstract domain vs other things?
%- Boolean Algebra
%- Cardinality/Value considerations
%- Forward analysis to be use as a subcomponent of other analyses
%- Abstract domain (implicit vs explicit control flow)
%- Inference (vs BAPA)
\section{Logic and Set Abstraction}
\label{sec:logic-and-set-abstraction} \label{s:3:abs}
We now define the elements and operators of a set abstract domain that meets the needs of all the analyses shown in Section~\ref{s:2:over}.

\paragraph{Concrete states.}
In this paper, we use symbols $\setvw$, $\setvx$, $\setvy$, and $\setvz$ as set variables
and let $\setvars$ represent the set of all such variables.
We are interested in purely symbolic set relations, and do not make any
assumption on the type of the set elements (in practice these are pointers
or scalars).
We let \( \values \) denote the set of all these elements.
A concrete state is a function \( \state: \setvars \rightarrow
\partsof{\values} \).
We write \( \states \) for the set of such elements.

\paragraph{Symbolic sets.}
Before we set up the signature of abstract domains, we fix a language
of set predicates, that will be used as a basis for abstract elements,
and for the communication with the set abstract domain.
\begin{definition}[Symbolic Sets]
  \label{d:1:symsets}
  {\em Symbolic sets} are defined by the grammar:
  \begin{align*}
    L (\in \symsets) ::=
    & \ L \wedge L \
    | \ E \subseteq E \
    | \ \card{X} = 1 \
    | \ \top \
    | \ \bot
    & E ::=
    & \ \emptyset \ | \ X \ | \ \comp{E} \ | \ E \cup E \ | \ E \suplus E
  \end{align*}
\end{definition}
The meaning of these constraints is straightforward, but we give a formal
definition in Figure~\ref{f:4:symsets} for clarity.
A model of a set expression $E$ is a concrete state $\state$ and a set
of concrete values $c$.
A model of a logical expression $L$ is a concrete state $\state$.
The concretization is $\gamma(L) = \{\ \sigma \ |\  \sigma \models L\ \}$
and we use $\aform{L}$ for abstract states with the same concretization.
\begin{figure}[t]
  \begin{align*}
    & \state, c \models \emptyset \textrm{ iff } c = \emptyset
    \qquad
    \state, c \models X \textrm{ iff } c = \state(X)
    \qquad \state, c \models \comp{E}
    \textrm{ iff }
    \state, c' \models{E}
    \textrm{ and } \forall v \in \values.
    \; v \in c \Leftrightarrow v \not\in c'
    \\
    & \state, c \models E_1 \cup E_2
    \textrm{ iff }
    \state, c_1 \models E_1
    \textrm{ and } \state, c_2 \models E_2
    \textrm{ and }
    \forall v \in \values. \; v \in c \Leftrightarrow v \in c_1 \vee
    v \in c_2
    \\
    & \state, c \models E_1 \suplus E_2
    \textrm{ iff }
    \state, c_1 \models E_1
    \textrm{ and } \state, c_2 \models E_2
    \textrm{ and }
    \forall v \in \values. \; v \in c \Leftrightarrow v \in c_1 \vee
    v \in c_2
    \textrm{ and } c_1 \cap c_2 = \emptyset
    \\
    & \state \models L_1 \wedge L_2
    \textrm{ iff }
    \state \models L_1 \textrm{ and } \state \models L_2
    \qquad
    \state \models \card{E} = 1
    \textrm{ iff }
    \state, c \models E \textrm{ and } \exists v \in \values. \; c = \{ v \}
    \\
    & \state \models E_1 \subseteq E_2
    \textrm{ iff }
    \state, c_1 \models E_1 \textrm{ and } \state, c_2 \models E_2
    \textrm{ and } \forall v \in \values. \; v \in c_1 \rightarrow v \in c_2
    \qquad \state \models \top
    \qquad \state \not\models \bot
  \end{align*}
  \caption{Symbolic set constraint language}
  \label{f:4:symsets}
\end{figure}
We shall also use the following derived logical forms for simplicity:
\[
E_1 \cap E_2 \defeq \comp{(\comp{E_1} \cup \comp{E_2})}
\qquad
E_1 = E_2 \defeq E_1 \subseteq E_2 \wedge E_2 \subseteq E_1
\qquad
E_1 \setminus E_2 \defeq E_1 \cap \comp{E_2}
\]

\paragraph{Set abstraction.}
A {\em set abstract domain} is defined by a set of {\em abstract elements}
\( \adom \) which describe the family of logical properties it can
express and a concretization function \( \gammadom: \adom \rightarrow
\partsof{\states} \) that maps each element of \( \adom \) into the set
of concrete states that satisfy it.
Abstract elements are characterized by
(1) the symbolic sets they describe and
(2) their machine representation.
The latter is usually very different from the formulas, and will be
discussed in Section~\ref{s:4:domains}.
\begin{example}[(Non-)Emptiness set domain]
  \label{ex:1:mt}
  A very basic example of such a domain is the {\em (non-)emptiness} domain
  that comprises the following elements:
  \begin{compactitem}
  \item \( \bot \), which denotes the unsatisfiable abstract constraint
    (\ie, \( \gammadom( \bot ) = \emptyset ) \));
  \item the functions from \( \setvars \) into \( \{ [=\emptyset],
    [\not=\emptyset], \top \} \), which map each set variable into
    its emptiness value.
  \end{compactitem}
  For instance, \( \{ \setvx \mapsto \top; \setvy \mapsto [=\emptyset] \} \)
  stands for \( \aform{\setvy \subseteq \emptyset} \) and concretizes into
  \( \gamma( \setvy \subseteq \emptyset ) \).
\end{example}

\paragraph{Operations over Set Abstractions.}
We now formalize the main operations and logical elements needed so that
we can use a set abstract \( \adom \) domain for either of the static
analyses shown in Section~\ref{s:2:over}.
\begin{asparaitem}
\item \emph{Basic logical elements.}
  Static analyses typically start with an unconstrained state.
  This is indicated by a \( \adomtop \in \adom \) element with full
  concretization, \ie, \( \gammadom( \adomtop ) = \states \).
  Similarly, the abstract element \( \adombot \in \adom \) should describe
  the unsatisfiable abstract constraint (\ie, \( \gammadom( \adombot ) =
  \emptyset \)).
  In Example~\ref{ex:1:mt}, \( \adombot \) is \( \bot \) and \( \adomtop \)
  is \( \lambda (x \in \setvars) \cdot \top \).
  Moreover, a static analysis often has to determine if an abstract state
  describes unsatisfiable constraints.
  Thus, \( \adom \) should provide an operator \( \adomisbot: \adom
  \rightarrow \{ \true, \false \} \) such that \( \adomisbot( \astate )
  = \true \Longrightarrow \gammadom( \astate ) = \emptyset \).

\item \emph{Forgetting a set variable.}
  Static analysis tools drop set variables that become redundant.
  In the open object example of Section~\ref{s:2:over}, this occurs when
  the singleton symbol is eliminated at the end of the loop.
  To do this, we require the set abstract domain \( \adom \) to provide
  an operator \( \adomforget: \adom \times \setvars \rightarrow \adom \)
  that discards a symbol from the abstract state.
  % forget soundness property

\item \emph{Assuming set constraints.}
  As noted in Section~\ref{s:2:over}, an important set reasoning step
  {\em restricts an abstract state with set constraints}, thus set
  domain \( \adom \) should provide an operator \( \adomassume: \adom
  \times \symsets \rightarrow \adom \), which conservatively represents
  a constraint into an abstract state, \ie ensures that, for all
  \( \astate, L \), \( \gammadom( \astate ) \cap \gamma( L ) \subseteq
  \gammadom( \adomassume( \astate, L ) ) \).
  Note that this operator also makes use of the symbolic set language
  of Definition~\ref{d:1:symsets} in order to describe constraints
  communicated to the domain.
  
\item \emph{Verifying set constraints.}
  Similarly, set reasoning should allow {\em verifying
    set constraints}, thus the set domain \( \adom \) should provide an operator
  \( \adomprove: \adom \times \symsets \rightarrow \{ \true, \false \} \),
  which conservatively attempts to verify that a symbolic set constraint
  holds under some abstract states, \ie ensures that, for all \( \astate,
  L \), \( \adomprove( \astate, L ) = \true \) implies that
  \( \gammadom( \astate ) \subseteq \gamma( L ) \).

\item \emph{Generalizing set abstractions.}
  The analysis of loops is commonly based on the computation of abstract
  post-fixpoints~\cite{cc:popl:77}, thus \( \adom \) should provide
  sound over-approximation of the union of sets concrete states.
  In the logical point of view, this amounts to computing a common weakening
  for two abstract constraints.
  This is performed by an operator \( \adomjoin: \adom \times \adom
  \rightarrow \adom \) such that, for all \( \astate_0, \astate_1 \),
  \( \gammadom( \astate_0 ) \cup \gammadom( \astate_1 ) \subseteq
  \gammadom( \adomjoin( \astate_0, \astate_1 ) ) \).
  Widening operator \( \adomwiden \) should satisfy the same property
  and ensure termination of any sequence of abstract iterates.

\item \emph{Deciding entailment over set abstractions.}
  Finally, the operator \( \adomisle: \adom \times \adom \rightarrow \{ \true,
  \false \} \) conservatively decides implication among abstract set
  constraints (by ensuring that \( \adomisle( \astate_0, \astate_1 ) =
  \true \Longrightarrow \gammadom( \astate_0 ) \subseteq \gammadom(
  \astate_1 ) \)), and allows verifying the convergence of abstract
  iterates.
\end{asparaitem}
%ac: Should we include the rename operator here?  It seems critical to a lot of what we do, and while we consider it fairly trivial for domains, it is a non-trivial, expensive operation for SMT and solver domains.


%4 Constructed set abstractions (5 pages)
%-> search for efficient structures and algorithms associated to them
%4.1 Lin
%4.2 QUICr
%4.3 BDD
%4.4 EQ functor
%4.5 Packing functor
%- Why packing doesn't work for BDDs in model checking, but does work here.
%
%5. Solver-based set abstractions (1.5 pages)
%-> encode problem to traditional, known problem
%5.1 SMT
%5.2 QBF
%
%
%6. Evaluation (2 pages)
%-> compare approaches in several classes of problems
%- Benchmark suites:
%- Python set tests
%- JSAna JavaScript verification
%- Memcad C data structure memory safety
%- Show:
%- Old set domains much slower
%- New set domains faster and often more precise
%- SMT doesn't scale in on these applications
%
%7. Tradeoffs/Limitations (0.5 pages)
%-> limitations and possible enhancements
%- Cardinality
%- Contents
%
%8. Related Work (1 page)
%
%9. Conclusions (0.5 pages)
    
%----------------------------------------------------------------------------
% Biblio.
\bibliographystyle{plain}
\bibliography{symbolic-sets}
\end{document}